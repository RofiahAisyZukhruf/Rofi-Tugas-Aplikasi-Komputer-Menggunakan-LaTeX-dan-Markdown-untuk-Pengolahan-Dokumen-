\documentclass{article}

\usepackage{eumat}

\begin{document}
\begin{eulernotebook}
\begin{eulercomment}
Nama  : Rofi'ah 'Aisy Zukhruf\\
NIM   : 23030630019\\
Kelas : Matematika B\\
\end{eulercomment}
\eulersubheading{}
\begin{eulercomment}
\begin{eulercomment}
\eulerheading{Menggambar Plot 3D dengan EMT}
\begin{eulercomment}
Ini adalah pengenalan plot 3D di Euler. Kami membutuhkan plot 3D untuk
memvisualisasikan fungsi dari dua variabel.

Euler menggambar fungsi seperti itu menggunakan algoritma pengurutan
untuk menyembunyikan bagian di latar belakang. Secara umum, Euler
menggunakan proyeksi pusat. Defaultnya adalah dari kuadran x-y positif
ke arah asal x = y = z = 0, tetapi sudut = 0 ? melihat dari arah sumbu
y. Sudut pandang dan ketinggian dapat diubah.

Euler bisa merencanakan

- permukaan dengan bayangan dan garis level atau rentang level,\\
- awan poin,\\
- kurva parametrik,\\
- permukaan implisit.

Plot 3D dari suatu fungsi menggunakan plot3d. Cara termudah adalah
dengan memplot ekspresi dalam x dan y. Parameter r mengatur kisaran
plot sekitar (0,0).
\end{eulercomment}
\begin{eulerprompt}
>aspect(1.5); plot3d("x^2+sin(y)",r=pi):
\end{eulerprompt}
\eulerimg{17}{images/Rofi'ah 'Aisy Zukhruf_23030630019_Matematika B_EMT4Plot3D-001.png}
\eulersubheading{Latihan Soal:}
\begin{eulercomment}
1. plot fungsi

\end{eulercomment}
\begin{eulerformula}
\[
z=x^2 + cos(y)
\]
\end{eulerformula}
\begin{eulercomment}
dalam rentang -3\textless{}=x\textless{}=3 dan -2pi\textless{}=y\textless{}=2pi. Gantilah sudut pandang untuk
memperlihatkan plot dengan lebih baik.

Jawab :
\end{eulercomment}
\begin{eulerprompt}
>aspect(1.5);
>plot3d("x^2+cos(y)",r=[-3, 3, -2*pi, 2*pi])
\end{eulerprompt}
\begin{euleroutput}
  Index 5 out of bounds!
  Try "trace errors" to inspect local variables after errors.
  plot3d:
      zmin=cz-r\{5\}; zmax=cz+r\{6\};
\end{euleroutput}
\begin{eulerprompt}
>view(30, 30):
\end{eulerprompt}
\eulerimg{17}{images/Rofi'ah 'Aisy Zukhruf_23030630019_Matematika B_EMT4Plot3D-003.png}
\begin{eulercomment}
2. Buatlah plot 3D dari fungsi \\
\end{eulercomment}
\begin{eulerformula}
\[
z=sin(x)+sin(y) 
\]
\end{eulerformula}
\begin{eulercomment}
pada rentang -p\textless{}=x\textless{}=p dan -pi\textless{}=y\textless{}=pi. Atur warna permukaan agar lebih
menarik.

JAWAB :
\end{eulercomment}
\begin{eulerprompt}
>aspect(1.5);
>plot3d("sin(x) + sin(y)", r = [-pi, pi, -pi, pi], color="blue");
\end{eulerprompt}
\begin{euleroutput}
  Index 5 out of bounds!
  Try "trace errors" to inspect local variables after errors.
  plot3d:
      zmin=cz-r\{5\}; zmax=cz+r\{6\};
\end{euleroutput}
\begin{eulerprompt}
>view(45, 20):
\end{eulerprompt}
\eulerimg{17}{images/Rofi'ah 'Aisy Zukhruf_23030630019_Matematika B_EMT4Plot3D-005.png}
\eulersubheading{}
\begin{eulercomment}
\begin{eulercomment}
\eulerheading{Fungsi Dua Variabel}
\begin{eulercomment}
Untuk grafik suatu fungsi, gunakan

- ekspresi sederhana dalam x dan y,\\
- nama fungsi dari dua variabel l\\
- atau matriks data.

Defaultnya adalah kisi kawat yang diisi dengan warna berbeda di kedua
sisi. Perhatikan bahwa jumlah default interval kisi adalah 10, tetapi
plot menggunakan jumlah default persegi panjang 40x40 untuk membuat
permukaan. Ini bisa diubah.

- n = 40, n = [40,40]: jumlah garis grid di setiap arah\\
- grid= 10, grid = [10,10]: jumlah garis kisi di setiap arah.

Kami menggunakan default n = 40 dan grid = 10.
\end{eulercomment}
\begin{eulerprompt}
>plot3d("x^2+y^2"):
\end{eulerprompt}
\eulerimg{17}{images/Rofi'ah 'Aisy Zukhruf_23030630019_Matematika B_EMT4Plot3D-006.png}
\begin{eulercomment}
Interaksi pengguna dimungkinkan dengan\textgreater{} user parameter. Pengguna dapat
menekan tombol berikut.

- left,right,up,down: putar sudut pandang\\
- +, -: memperbesar atau memperkecil\\
- a: menghasilkan anaglyph (lihat di bawah)\\
- l: beralih memutar sumber cahaya (lihat di bawah)\\
- space: reset ke default\\
- return: interaksi akhir
\end{eulercomment}
\begin{eulerprompt}
>plot3d("exp(-x^2+y^2)",>user, ...
>  title="Turn with the vector keys (press return to finish)"):
\end{eulerprompt}
\eulerimg{17}{images/Rofi'ah 'Aisy Zukhruf_23030630019_Matematika B_EMT4Plot3D-007.png}
\begin{eulercomment}
Rentang plot untuk fungsi dapat ditentukan dengan

- a, b: rentang-x\\
- c, d: rentang y\\
- r: persegi simetris di sekitar (0,0).\\
- n: jumlah subinterval untuk plot.

Ada beberapa parameter untuk menskalakan fungsi atau mengubah tampilan
grafik.

fscale: menskalakan ke nilai fungsi (defaultnya adalah \textless{}fscale).\\
scale: angka atau vektor 1x2 untuk skala ke arah x dan y.\\
frame: jenis bingkai (default 1)
\end{eulercomment}
\begin{eulerprompt}
>plot3d("exp(-(x^2+y^2)/5)",r=10,n=80,fscale=4,scale=1.2,frame=3):
\end{eulerprompt}
\eulerimg{17}{images/Rofi'ah 'Aisy Zukhruf_23030630019_Matematika B_EMT4Plot3D-008.png}
\begin{eulercomment}
Tampilan dapat diubah dengan berbagai cara.

- distance: jarak pandang ke plot.\\
- zoom: nilai zoom.\\
- angle: sudut ke sumbu y negatif dalam radian.\\
- height: ketinggian tampilan dalam radian.

Nilai default bisa diperiksa atau diubah dengan fungsi view (). Ini
mengembalikan parameter dalam urutan di atas.
\end{eulercomment}
\begin{eulerprompt}
>view
\end{eulerprompt}
\begin{euleroutput}
  [5,  2.6,  2,  0.4]
\end{euleroutput}
\begin{eulercomment}
Jarak yang lebih dekat membutuhkan lebih sedikit zoom. Efeknya lebih
seperti lensa sudut lebar.

Dalam contoh berikut, sudut = 0 dan tinggi = 0 dilihat dari sumbu y
negatif. Label sumbu untuk y disembunyikan dalam kasus ini.
\end{eulercomment}
\begin{eulerprompt}
>plot3d("x^2+y",distance=3,zoom=2,angle=0,height=0):
\end{eulerprompt}
\eulerimg{17}{images/Rofi'ah 'Aisy Zukhruf_23030630019_Matematika B_EMT4Plot3D-009.png}
\begin{eulercomment}
Plot selalu terlihat ke tengah plot kubus. Anda dapat memindahkan
pusat dengan parameter tengah.
\end{eulercomment}
\begin{eulerprompt}
>plot3d("x^4+y^2",a=0,b=1,c=-1,d=1,angle=-20°,height=20°, ...
>  center=[0.4,0,0],zoom=5):
\end{eulerprompt}
\eulerimg{17}{images/Rofi'ah 'Aisy Zukhruf_23030630019_Matematika B_EMT4Plot3D-010.png}
\begin{eulercomment}
Plot diskalakan agar sesuai dengan kubus satuan untuk dilihat. Jadi
tidak perlu mengubah jarak atau zoom tergantung ukuran plot. Namun,
label mengacu pada ukuran sebenarnya.

Jika Anda mematikannya dengan scale = false, Anda harus berhati-hati,
bahwa plot masih pas dengan jendela plotting, dengan mengubah jarak
pandang atau zoom, dan memindahkan pusat.
\end{eulercomment}
\begin{eulerprompt}
>plot3d("5*exp(-x^2-y^2)",r=2,<fscale,<scale,distance=13,height=50°, ...
>  center=[0,0,-2],frame=3):
\end{eulerprompt}
\eulerimg{17}{images/Rofi'ah 'Aisy Zukhruf_23030630019_Matematika B_EMT4Plot3D-011.png}
\begin{eulercomment}
Plot kutub juga tersedia. The parameter polar= true menggambarkan plot
kutub. Fungsi tersebut harus tetap merupakan fungsi dari x dan y.
Parameter "fscale" menskalakan fungsi dengan skala sendiri. Jika
tidak, fungsi diskalakan agar sesuai dengan kubus.
\end{eulercomment}
\begin{eulerprompt}
>plot3d("1/(x^2+y^2+1)",r=5,>polar, ...
>fscale=2,>hue,n=100,zoom=4,>contour,color=gray):
\end{eulerprompt}
\eulerimg{17}{images/Rofi'ah 'Aisy Zukhruf_23030630019_Matematika B_EMT4Plot3D-012.png}
\begin{eulerprompt}
>function f(r) := exp(-r/2)*cos(r); ...
>plot3d("f(x^2+y^2)",>polar,scale=[1,1,0.4],r=2pi,frame=3,zoom=4):
\end{eulerprompt}
\eulerimg{17}{images/Rofi'ah 'Aisy Zukhruf_23030630019_Matematika B_EMT4Plot3D-013.png}
\begin{eulercomment}
Parameter memutar memutar fungsi dalam x di sekitar sumbu x.

- rotate = 1: Menggunakan sumbu x\\
- rotate = 2: Menggunakan sumbu z
\end{eulercomment}
\begin{eulerprompt}
>plot3d("x^2+1",a=-1,b=1,rotate=true,grid=5):
\end{eulerprompt}
\eulerimg{17}{images/Rofi'ah 'Aisy Zukhruf_23030630019_Matematika B_EMT4Plot3D-014.png}
\begin{eulercomment}
Berikut adalah plot dengan tiga fungsi.
\end{eulercomment}
\begin{eulerprompt}
>plot3d("x","x^2+y^2","y",r=2,zoom=3.5,frame=3):
\end{eulerprompt}
\eulerimg{17}{images/Rofi'ah 'Aisy Zukhruf_23030630019_Matematika B_EMT4Plot3D-015.png}
\eulersubheading{Latihan Soal}
\begin{eulercomment}
1. Plot fungsi\\
\end{eulercomment}
\begin{eulerformula}
\[
z=ln(x^2+y^2+1) 
\]
\end{eulerformula}
\begin{eulercomment}
untuk rentang 0\textless{}x\textless{}2 dan -2\textless{}y\textless{}2. Atur jumlah subinterval untuk plot
menjadi 50.

JAWAB :
\end{eulercomment}
\begin{eulerprompt}
>aspect(1.5);
>plot3d("ln(x^2 + y^2 + 1)", a=0, b=2, c=-2, d=2, n=50, frame=1):
\end{eulerprompt}
\eulerimg{17}{images/Rofi'ah 'Aisy Zukhruf_23030630019_Matematika B_EMT4Plot3D-017.png}
\begin{eulercomment}
2. Gambarlah permukaan dari fungsi\\
\end{eulercomment}
\begin{eulerformula}
\[
z=x^3-3xy^2 
\]
\end{eulerformula}
\begin{eulercomment}
dalam rentang 2\textless{}=x\textless{}=2 dan 2\textless{}=y\textless{}=2. Gantilah sudut pandang untuk
memberikan tampilan yang lebih baik.

JAWAB :
\end{eulercomment}
\begin{eulerprompt}
>aspect(1.5);
>plot3d("x^3 - 3*x*y^2", r=2, n=40);
\end{eulerprompt}
\begin{euleroutput}
  
  Space between commands expected!
  Found: 
   (character 10)
  You can disable this in the Options menu.
  Error in:
  plot3d("x^3 - 3*x*y^2", r=2, n=40);
   ...
                                     ^
\end{euleroutput}
\begin{eulerprompt}
>view(45, 30):
\end{eulerprompt}
\eulerimg{17}{images/Rofi'ah 'Aisy Zukhruf_23030630019_Matematika B_EMT4Plot3D-019.png}
\eulersubheading{}
\begin{eulercomment}
\begin{eulercomment}
\eulerheading{Plot Kontur}
\begin{eulercomment}
Untuk plot, Euler menambahkan garis kisi. Alih-alih, dimungkinkan
untuk menggunakan garis level dan corak satu warna atau corak warna
spektral. Euler dapat menggambar ketinggian fungsi pada plot dengan
shading. Di semua plot 3D, Euler dapat menghasilkan anaglyph merah /
cyan.

-\textgreater{} hue: Mengaktifkan bayangan terang, bukan kabel.\\
-\textgreater{} contour: Membuat plot garis kontur otomatis pada plot.\\
- level = ... (atau level): Vektor nilai untuk garis kontur.

Standarnya adalah level = "auto", yang menghitung beberapa baris level
secara otomatis. Seperti yang Anda lihat di plot, level sebenarnya
adalah rentang level.

Gaya default dapat diubah. Untuk plot kontur berikut, kami menggunakan
kisi yang lebih halus untuk 100x100 titik, menskalakan fungsi dan
plot, dan menggunakan sudut pandang yang berbeda.
\end{eulercomment}
\begin{eulerprompt}
>plot3d("exp(-x^2-y^2)",r=2,n=100,level="thin", ...
> >contour,>spectral,fscale=1,scale=1.1,angle=45°,height=20°):
\end{eulerprompt}
\eulerimg{17}{images/Rofi'ah 'Aisy Zukhruf_23030630019_Matematika B_EMT4Plot3D-020.png}
\begin{eulerprompt}
>plot3d("exp(x*y)",angle=100°,>contour,color=green):
\end{eulerprompt}
\eulerimg{17}{images/Rofi'ah 'Aisy Zukhruf_23030630019_Matematika B_EMT4Plot3D-021.png}
\begin{eulercomment}
Bayangan default menggunakan warna abu-abu. Tetapi berbagai spektrum
warna juga tersedia.

-\textgreater{} spectral: Menggunakan skema spektral default\\
- color = ...: Menggunakan warna khusus atau skema spektral

Untuk plot berikut, kami menggunakan skema spektral default dan
menambah jumlah poin untuk mendapatkan tampilan yang sangat mulus.
\end{eulercomment}
\begin{eulerprompt}
>plot3d("x^2+y^2",>spectral,>contour,n=100):
\end{eulerprompt}
\eulerimg{17}{images/Rofi'ah 'Aisy Zukhruf_23030630019_Matematika B_EMT4Plot3D-022.png}
\begin{eulercomment}
Alih-alih garis level otomatis, kami juga dapat mengatur nilai garis
level. Ini akan menghasilkan garis level tipis, bukan rentang level.
\end{eulercomment}
\begin{eulerprompt}
>plot3d("x^2-y^2",0,1,0,1,angle=220°,level=-1:0.2:1,color=redgreen):
\end{eulerprompt}
\eulerimg{17}{images/Rofi'ah 'Aisy Zukhruf_23030630019_Matematika B_EMT4Plot3D-023.png}
\begin{eulercomment}
Dalam plot berikut, kami menggunakan dua pita level yang sangat luas
dari -0,1 hingga 1, dan dari 0,9 hingga 1. Ini dimasukkan sebagai
matriks dengan batas level sebagai kolom.

Selain itu, kami melapisi kisi dengan 10 interval di setiap arah.
\end{eulercomment}
\begin{eulerprompt}
>plot3d("x^2+y^3",level=[-0.1,0.9;0,1], ...
>  >spectral,angle=30°,grid=10,contourcolor=gray):
\end{eulerprompt}
\eulerimg{17}{images/Rofi'ah 'Aisy Zukhruf_23030630019_Matematika B_EMT4Plot3D-024.png}
\begin{eulercomment}
Dalam contoh berikut, kami memplot set, di mana

\end{eulercomment}
\begin{eulerformula}
\[
f(x,y) = x^y-y^x = 0
\]
\end{eulerformula}
\begin{eulercomment}
Kami menggunakan satu garis tipis untuk garis level.
\end{eulercomment}
\begin{eulerprompt}
>plot3d("x^y-y^x",level=0,a=0,b=6,c=0,d=6,contourcolor=red,n=100):
\end{eulerprompt}
\eulerimg{17}{images/Rofi'ah 'Aisy Zukhruf_23030630019_Matematika B_EMT4Plot3D-025.png}
\begin{eulercomment}
Dimungkinkan untuk menunjukkan bidang kontur di bawah plot. Warna dan
jarak ke plot dapat ditentukan.
\end{eulercomment}
\begin{eulerprompt}
>plot3d("x^2+y^4",>cp,cpcolor=green,cpdelta=0.2):
\end{eulerprompt}
\eulerimg{17}{images/Rofi'ah 'Aisy Zukhruf_23030630019_Matematika B_EMT4Plot3D-026.png}
\begin{eulercomment}
Berikut beberapa gaya lainnya. Kami selalu mematikan bingkai, dan
menggunakan berbagai skema warna untuk plot dan kisi.
\end{eulercomment}
\begin{eulerprompt}
>figure(2,2); ...
>expr="y^3-x^2"; ...
>figure(1);  ...
>  plot3d(expr,<frame,>cp,cpcolor=spectral); ...
>figure(2);  ...
>  plot3d(expr,<frame,>spectral,grid=10,cp=2); ...
>figure(3);  ...
>  plot3d(expr,<frame,>contour,color=gray,nc=5,cp=3,cpcolor=greenred); ...
>figure(4);  ...
>  plot3d(expr,<frame,>hue,grid=10,>transparent,>cp,cpcolor=gray); ...
>figure(0):
\end{eulerprompt}
\eulerimg{17}{images/Rofi'ah 'Aisy Zukhruf_23030630019_Matematika B_EMT4Plot3D-027.png}
\begin{eulercomment}
Ada beberapa skema spektral lain, diberi nomor dari 1 hingga 9. Tetapi
Anda juga dapat menggunakan color=value, di mana nilai

- spectral: untuk rentang dari biru hingga merah\\
- white: untuk rentang yang lebih redup\\
- yellowblue,purplegreen,blueyellow,greenred\\
- blueyellow, greenpurple,yellowblue,redgreen
\end{eulercomment}
\begin{eulerprompt}
>figure(3,3); ...
>for i=1:9;  ...
>  figure(i); plot3d("x^2+y^2",spectral=i,>contour,>cp,<frame,zoom=4);  ...
>end; ...
>figure(0):
\end{eulerprompt}
\eulerimg{17}{images/Rofi'ah 'Aisy Zukhruf_23030630019_Matematika B_EMT4Plot3D-028.png}
\begin{eulercomment}
Sumber cahaya dapat diubah dengan l dan tombol kursor selama interaksi
pengguna. Itu juga dapat diatur dengan parameter.

-light : arah cahaya\\
- amb: cahaya ambient antara 0 dan 1

Perhatikan bahwa program tidak membuat perbedaan antara sisi-sisi
plot. Tidak ada bayangan. Untuk ini, Anda membutuhkan Povray.
\end{eulercomment}
\begin{eulerprompt}
>plot3d("-x^2-y^2", ...
>  hue=true,light=[0,1,1],amb=0,user=true, ...
>  title="Press l and cursor keys (return to exit)"):
\end{eulerprompt}
\eulerimg{17}{images/Rofi'ah 'Aisy Zukhruf_23030630019_Matematika B_EMT4Plot3D-029.png}
\begin{eulercomment}
Parameter warna mengubah warna permukaan. Warna garis level juga dapat
diubah.
\end{eulercomment}
\begin{eulerprompt}
>plot3d("-x^2-y^2",color=rgb(0.2,0.2,0),hue=true,frame=false, ...
>  zoom=3,contourcolor=red,level=-2:0.1:1,dl=0.01):
\end{eulerprompt}
\eulerimg{17}{images/Rofi'ah 'Aisy Zukhruf_23030630019_Matematika B_EMT4Plot3D-030.png}
\begin{eulercomment}
Warna 0 memberikan efek pelangi khusus.
\end{eulercomment}
\begin{eulerprompt}
>plot3d("x^2/(x^2+y^2+1)",color=0,hue=true,grid=10):
\end{eulerprompt}
\eulerimg{17}{images/Rofi'ah 'Aisy Zukhruf_23030630019_Matematika B_EMT4Plot3D-031.png}
\begin{eulercomment}
Permukaannya juga bisa transparan.
\end{eulercomment}
\begin{eulerprompt}
>plot3d("x^2+y^2",>transparent,grid=10,wirecolor=red):
\end{eulerprompt}
\eulerimg{17}{images/Rofi'ah 'Aisy Zukhruf_23030630019_Matematika B_EMT4Plot3D-032.png}
\eulersubheading{Latihan Soal}
\begin{eulercomment}
1. Plot fungsi\\
\end{eulercomment}
\begin{eulerformula}
\[
z=e^(-x^2-y^2)
\]
\end{eulerformula}
\begin{eulercomment}
dengan garis kontur otomatis. Gunakan rentang -2\textless{}=x\textless{}=2 dan -2\textless{}=y\textless{}=2.
Atur jumlah titik untuk mendapatkan tampilan yang halus dengan 100x100
titik.
\end{eulercomment}
\begin{eulerprompt}
>aspect(1.5);
>plot3d("exp(-x^2 - y^2)", r=2, n=100, <contour, <spectral):
\end{eulerprompt}
\eulerimg{17}{images/Rofi'ah 'Aisy Zukhruf_23030630019_Matematika B_EMT4Plot3D-034.png}
\begin{eulercomment}
2. Gambarlah garis kontur dari fungsi\\
\end{eulercomment}
\begin{eulerformula}
\[
z=x^2-y^2
\]
\end{eulerformula}
\begin{eulercomment}
dengan level garis kontur yang ditentukan dari -1 hingga 1 dengan
interval 0.2. Gunakan rentang 1\textless{}=x\textless{}=1 dan 1\textless{}=y\textless{}=1.

JAWAB :
\end{eulercomment}
\begin{eulerprompt}
>aspect(1.5);
>plot3d("x^2 - y^2", a=-1, b=1, c=-1, d=1, level=-1:0.2:1, <contour, color=redgreen):
\end{eulerprompt}
\eulerimg{17}{images/Rofi'ah 'Aisy Zukhruf_23030630019_Matematika B_EMT4Plot3D-036.png}
\eulersubheading{}
\begin{eulercomment}
\begin{eulercomment}
\eulerheading{Plot Implisit}
\begin{eulercomment}
Ada juga plot implisit dalam tiga dimensi. Euler menghasilkan
pemotongan melalui objek. Fitur plot3d termasuk plot implisit. Plot
ini menunjukkan himpunan nol fungsi dalam tiga variabel.\\
Solusi dari

\end{eulercomment}
\begin{eulerformula}
\[
f(x,y,z) = 0
\]
\end{eulerformula}
\begin{eulercomment}
dapat divisualisasikan dalam potongan sejajar dengan bidang x-y-,
x-z-, dan y-z.

- implisit = 1: potong sejajar bidang y-z\\
- implisit = 2: potong sejajar dengan bidang x-z\\
- implisit = 4: potong sejajar bidang x-y

Tambahkan nilai-nilai ini, jika Anda suka. Dalam contoh yang kami plot

\end{eulercomment}
\begin{eulerformula}
\[
M = \{ (x,y,z) : x^2+y^3+zy=1 \}
\]
\end{eulerformula}
\begin{eulerprompt}
>plot3d("x^2+y^3+z*y-1",r=5,implicit=3):
\end{eulerprompt}
\eulerimg{17}{images/Rofi'ah 'Aisy Zukhruf_23030630019_Matematika B_EMT4Plot3D-039.png}
\begin{eulerprompt}
>c=1; d=1;
>plot3d("((x^2+y^2-c^2)^2+(z^2-1)^2)*((y^2+z^2-c^2)^2+(x^2-1)^2)*((z^2+x^2-c^2)^2+(y^2-1)^2)-d",r=2,<frame,>implicit,>user): 
\end{eulerprompt}
\eulerimg{17}{images/Rofi'ah 'Aisy Zukhruf_23030630019_Matematika B_EMT4Plot3D-040.png}
\begin{eulerprompt}
>plot3d("x^2+y^2+4*x*z+z^3",>implicit,r=2,zoom=2.5):
\end{eulerprompt}
\eulerimg{17}{images/Rofi'ah 'Aisy Zukhruf_23030630019_Matematika B_EMT4Plot3D-041.png}
\eulersubheading{Soal Latihan}
\begin{eulercomment}
1. Plot himpunan nol dari fungsi\\
\end{eulercomment}
\begin{eulerformula}
\[
f(x,y,z)=x^2+y^2+z^2-4. 
\]
\end{eulerformula}
\begin{eulercomment}
Gunakan potongan sejajar dengan bidang x-y dan x-z.

JAWAB :
\end{eulercomment}
\begin{eulerprompt}
>plot3d("x^2 + y^2 + z^2 - 4", r=3, implicit=3):
\end{eulerprompt}
\eulerimg{17}{images/Rofi'ah 'Aisy Zukhruf_23030630019_Matematika B_EMT4Plot3D-043.png}
\begin{eulercomment}
2. Gambarlah potongan dari himpunan nol fungsi\\
\end{eulercomment}
\begin{eulerformula}
\[
f(x,y,z)=x^2+y^2+z-1 
\]
\end{eulerformula}
\begin{eulercomment}
dengan potongan sejajar bidang y-z dan x-z.

JAWAB :
\end{eulercomment}
\begin{eulerprompt}
>plot3d("x^2 + y^3 + z - 1", r=2, implicit=6):
\end{eulerprompt}
\eulerimg{17}{images/Rofi'ah 'Aisy Zukhruf_23030630019_Matematika B_EMT4Plot3D-045.png}
\eulersubheading{}
\begin{eulercomment}
\begin{eulercomment}
\eulerheading{Plotting 3D Data}
\begin{eulercomment}
Sama seperti plot2d, plot3d menerima data. Untuk objek 3D, Anda perlu
memberikan matriks nilai x, y dan z, atau tiga fungsi atau ekspresi fx
(x, y), fy (x, y), fz (x, y).

\end{eulercomment}
\begin{eulerformula}
\[
\gamma(t,s) = (x(t,s),y(t,s),z(t,s))
\]
\end{eulerformula}
\begin{eulercomment}
Karena x, y, z adalah matriks, kita asumsikan bahwa (t, s) berjalan
melalui kotak persegi. Hasilnya, Anda dapat memplot gambar persegi
panjang di luar angkasa.

Anda dapat menggunakan bahasa matriks Euler untuk menghasilkan
koordinat secara efektif.

Dalam contoh berikut, kami menggunakan vektor nilai t dan vektor kolom
nilai s untuk membuat parameter permukaan bola. Dalam gambar kita bisa
menandai daerah, dalam kasus kita daerah kutub.
\end{eulercomment}
\begin{eulerprompt}
>t=linspace(0,2pi,180); s=linspace(-pi/2,pi/2,90)'; ...
>x=cos(s)*cos(t); y=cos(s)*sin(t); z=sin(s); ...
>plot3d(x,y,z,>hue, ...
>color=blue,<frame,grid=[10,20], ...
>values=s,contourcolor=red,level=[90°-24°;90°-22°], ...
>scale=1.4,height=50°):
\end{eulerprompt}
\eulerimg{17}{images/Rofi'ah 'Aisy Zukhruf_23030630019_Matematika B_EMT4Plot3D-047.png}
\begin{eulercomment}
Berikut adalah contoh grafik dari suatu fungsi.
\end{eulercomment}
\begin{eulerprompt}
>t=-1:0.1:1; s=(-1:0.1:1)'; plot3d(t,s,t*s,grid=10):
\end{eulerprompt}
\eulerimg{17}{images/Rofi'ah 'Aisy Zukhruf_23030630019_Matematika B_EMT4Plot3D-048.png}
\begin{eulercomment}
Namun, kita bisa membuat semua jenis permukaan. Ini adalah permukaan
yang sama sebagai suatu fungsi

\end{eulercomment}
\begin{eulerformula}
\[
x = y \, z
\]
\end{eulerformula}
\begin{eulerprompt}
>plot3d(t*s,t,s,angle=180°,grid=10):
\end{eulerprompt}
\eulerimg{17}{images/Rofi'ah 'Aisy Zukhruf_23030630019_Matematika B_EMT4Plot3D-049.png}
\begin{eulercomment}
Dengan lebih banyak usaha, kami dapat menghasilkan banyak permukaan.

Dalam contoh berikut, kami membuat tampilan berbayang dari bola yang
terdistorsi. Koordinat biasa untuk bola adalah

\end{eulercomment}
\begin{eulerformula}
\[
\gamma(t,s) = (\cos(t)\cos(s),\sin(t)\sin(s),\cos(s))
\]
\end{eulerformula}
\begin{eulercomment}
dengan

\end{eulercomment}
\begin{eulerformula}
\[
0 \le t \le 2\pi, \quad \frac{-\pi}{2} \le s \le \frac{\pi}{2}.
\]
\end{eulerformula}
\begin{eulercomment}
Kami menyimpangkan ini dengan sebuah faktor

\end{eulercomment}
\begin{eulerformula}
\[
d(t,s) = \frac{\cos(4t)+\cos(8s)}{4}.
\]
\end{eulerformula}
\begin{eulerprompt}
>t=linspace(0,2pi,320); s=linspace(-pi/2,pi/2,160)'; ...
>d=1+0.2*(cos(4*t)+cos(8*s)); ...
>plot3d(cos(t)*cos(s)*d,sin(t)*cos(s)*d,sin(s)*d,hue=1, ...
>  light=[1,0,1],frame=0,zoom=5):
\end{eulerprompt}
\eulerimg{17}{images/Rofi'ah 'Aisy Zukhruf_23030630019_Matematika B_EMT4Plot3D-050.png}
\begin{eulercomment}
Tentu saja, point cloud juga dimungkinkan. Untuk memplot data titik
dalam ruang, kita membutuhkan tiga vektor sebagai koordinat titik.

Gayanya sama seperti di plot2d dengan points = true;
\end{eulercomment}
\begin{eulerprompt}
>n=500;  ...
>  plot3d(normal(1,n),normal(1,n),normal(1,n),points=true,style="."):
\end{eulerprompt}
\eulerimg{17}{images/Rofi'ah 'Aisy Zukhruf_23030630019_Matematika B_EMT4Plot3D-051.png}
\begin{eulercomment}
Juga dimungkinkan untuk memplot kurva dalam 3D. Dalam kasus ini, lebih
mudah untuk menghitung sebelumnya titik-titik kurva. Untuk kurva di
bidang kita menggunakan urutan koordinat dan parameter wire = true.
\end{eulercomment}
\begin{eulerprompt}
>t=linspace(0,8pi,500); ...
>plot3d(sin(t),cos(t),t/10,>wire,zoom=3):
\end{eulerprompt}
\eulerimg{17}{images/Rofi'ah 'Aisy Zukhruf_23030630019_Matematika B_EMT4Plot3D-052.png}
\begin{eulerprompt}
>t=linspace(0,4pi,1000); plot3d(cos(t),sin(t),t/2pi,>wire, ...
>linewidth=3,wirecolor=blue):
\end{eulerprompt}
\eulerimg{17}{images/Rofi'ah 'Aisy Zukhruf_23030630019_Matematika B_EMT4Plot3D-053.png}
\begin{eulerprompt}
>X=cumsum(normal(3,100)); ...
> plot3d(X[1],X[2],X[3],>anaglyph,>wire):
\end{eulerprompt}
\eulerimg{17}{images/Rofi'ah 'Aisy Zukhruf_23030630019_Matematika B_EMT4Plot3D-054.png}
\begin{eulercomment}
EMT juga dapat memplot dalam mode anaglyph. Untuk melihat plot seperti
itu, Anda membutuhkan red/cyan glasses.
\end{eulercomment}
\begin{eulerprompt}
> plot3d("x^2+y^3",>anaglyph,>contour,angle=30°):
\end{eulerprompt}
\eulerimg{17}{images/Rofi'ah 'Aisy Zukhruf_23030630019_Matematika B_EMT4Plot3D-055.png}
\begin{eulercomment}
Seringkali, skema warna spektral digunakan untuk plot. Ini menekankan
ketinggian fungsinya.
\end{eulercomment}
\begin{eulerprompt}
>plot3d("x^2*y^3-y",>spectral,>contour,zoom=3.2):
\end{eulerprompt}
\eulerimg{17}{images/Rofi'ah 'Aisy Zukhruf_23030630019_Matematika B_EMT4Plot3D-056.png}
\begin{eulercomment}
Euler juga dapat memplot permukaan berparameter, jika parameternya
adalah nilai x, y, dan z dari gambar kisi persegi panjang di dalam
ruang.

Untuk demo berikut, kami menyiapkan parameter u- dan v-, dan
menghasilkan koordinat ruang dari ini.
\end{eulercomment}
\begin{eulerprompt}
>u=linspace(-1,1,10); v=linspace(0,2*pi,50)'; ...
>X=(3+u*cos(v/2))*cos(v); Y=(3+u*cos(v/2))*sin(v); Z=u*sin(v/2); ...
>plot3d(X,Y,Z,>anaglyph,<frame,>wire,scale=2.3):
\end{eulerprompt}
\eulerimg{17}{images/Rofi'ah 'Aisy Zukhruf_23030630019_Matematika B_EMT4Plot3D-057.png}
\begin{eulercomment}
Berikut adalah contoh yang lebih rumit, yang megah dengan red/cyan
glasses.
\end{eulercomment}
\begin{eulerprompt}
>u:=linspace(-pi,pi,160); v:=linspace(-pi,pi,400)';  ...
>x:=(4*(1+.25*sin(3*v))+cos(u))*cos(2*v); ...
>y:=(4*(1+.25*sin(3*v))+cos(u))*sin(2*v); ...
> z=sin(u)+2*cos(3*v); ...
>plot3d(x,y,z,frame=0,scale=1.5,hue=1,light=[1,0,-1],zoom=2.8,>anaglyph):
\end{eulerprompt}
\eulerimg{17}{images/Rofi'ah 'Aisy Zukhruf_23030630019_Matematika B_EMT4Plot3D-058.png}
\eulersubheading{}
\begin{eulercomment}
\end{eulercomment}
\eulersubheading{Soal Latihan}
\begin{eulercomment}
1. Buatlah plot permukaan dari fungsi\\
\end{eulercomment}
\begin{eulerformula}
\[
z=sin(x)*cos(y)
\]
\end{eulerformula}
\begin{eulercomment}
untuk rentang -pi\textless{}=x\textless{}=pi dan -pi\textless{}=y\textless{}=pi dengan grid 50x50.

JAWAB :
\end{eulercomment}
\begin{eulerprompt}
>x := linspace(-pi, pi, 50);
>y := linspace(-pi, pi, 50)';
\end{eulerprompt}
\begin{euleroutput}
  
  Space between commands expected!
  Found: 
   (character 10)
  You can disable this in the Options menu.
  Error in:
  y := linspace(-pi, pi, 50)';
   ...
                              ^
\end{euleroutput}
\begin{eulerprompt}
>z = sin(x) * cos(y);
>plot3d(x, y, z, hue=0, frame=0, grid=[10, 20]):
\end{eulerprompt}
\eulerimg{17}{images/Rofi'ah 'Aisy Zukhruf_23030630019_Matematika B_EMT4Plot3D-060.png}
\begin{eulercomment}
2. Gambarlah plot untuk kurva 3D dengan parameter x(t)=cos(t),
y(t)=sin(t), dan z(t)=t dengan t dari 0 hingga 4pi.

JAWAB :
\end{eulercomment}
\begin{eulerprompt}
>t = linspace(0, 4 * pi, 1000);
>x = cos(t); y = sin(t); z = t;
>plot3d(x, y, z, wire=true, linewidth=2, wirecolor=blue):
\end{eulerprompt}
\eulerimg{17}{images/Rofi'ah 'Aisy Zukhruf_23030630019_Matematika B_EMT4Plot3D-061.png}
\eulersubheading{}
\begin{eulercomment}
\begin{eulercomment}
\eulerheading{Plot Statistik}
\begin{eulercomment}
Plot batang juga dimungkinkan. Untuk ini, kami harus menyediakan

- x: vektor baris dengan n + 1 elemen\\
- y: vektor kolom dengan n + 1 elemen\\
- z: nxn matriks nilai.

z bisa lebih besar, tetapi hanya nilai nxn yang akan digunakan.

Dalam contoh, pertama-tama kita menghitung nilainya. Kemudian kita
menyesuaikan x dan y, sehingga vektor berpusat pada nilai yang
digunakan.
\end{eulercomment}
\begin{eulerprompt}
>x=-1:0.1:1; y=x'; z=x^2+y^2; ...
>xa=(x|1.1)-0.05; ya=(y_1.1)-0.05; ...
>plot3d(xa,ya,z,bar=true):
\end{eulerprompt}
\eulerimg{17}{images/Rofi'ah 'Aisy Zukhruf_23030630019_Matematika B_EMT4Plot3D-062.png}
\begin{eulercomment}
Dimungkinkan untuk membagi plot permukaan menjadi dua bagian atau
lebih.
\end{eulercomment}
\begin{eulerprompt}
>x=-1:0.1:1; y=x'; z=x+y; d=zeros(size(x)); ...
>plot3d(x,y,z,disconnect=2:2:20):
\end{eulerprompt}
\eulerimg{17}{images/Rofi'ah 'Aisy Zukhruf_23030630019_Matematika B_EMT4Plot3D-063.png}
\begin{eulercomment}
Jika memuat atau membuat matriks data M dari file dan perlu memplotnya
dalam 3D, Anda dapat menskalakan matriks ke [-1,1] dengan skala (M),
atau menskalakan matriks dengan\textgreater{} zscale. Ini dapat dikombinasikan
dengan faktor penskalaan individu yang diterapkan sebagai tambahan.
\end{eulercomment}
\begin{eulerprompt}
>i=1:20; j=i'; ...
>plot3d(i*j^2+100*normal(20,20),>zscale,scale=[1,1,1.5],angle=-40°,zoom=1.8):
\end{eulerprompt}
\eulerimg{17}{images/Rofi'ah 'Aisy Zukhruf_23030630019_Matematika B_EMT4Plot3D-064.png}
\begin{eulerprompt}
>Z=intrandom(5,100,6); v=zeros(5,6); ...
>loop 1 to 5; v[#]=getmultiplicities(1:6,Z[#]); end; ...
>columnsplot3d(v',scols=1:5,ccols=[1:5]):
\end{eulerprompt}
\eulerimg{17}{images/Rofi'ah 'Aisy Zukhruf_23030630019_Matematika B_EMT4Plot3D-065.png}
\eulersubheading{}
\begin{eulercomment}
\end{eulercomment}
\eulersubheading{Latihan Soal}
\begin{eulercomment}
1. Gambarlah plot batang 3D dari fungsi\\
\end{eulercomment}
\begin{eulerformula}
\[
z=x^2+y^2
\]
\end{eulerformula}
\begin{eulercomment}
untuk x dan y dalam rentang dari -1 hingga 1 dengan interval 0.1.

JAWAB :
\end{eulercomment}
\begin{eulerprompt}
>x = -1:0.1:1; y = x'; z = x^2 + y^2;
>xa = (x|1.1) - 0.05; ya = (y_1.1) - 0.05;
>plot3d(xa, ya, z, bar=true):
\end{eulerprompt}
\eulerimg{17}{images/Rofi'ah 'Aisy Zukhruf_23030630019_Matematika B_EMT4Plot3D-067.png}
\begin{eulercomment}
2. Buatlah plot permukaan dari fungsi Z=x+y dan pisahkan plot menjadi
dua bagian untuk x dan y dalam rentang dari -1 hingga 1.

JAWAB :
\end{eulercomment}
\begin{eulerprompt}
>x = -1:0.1:1; y = x';
>z = x + y;
>plot3d(x, y, z, disconnect=2:2:20):
\end{eulerprompt}
\eulerimg{17}{images/Rofi'ah 'Aisy Zukhruf_23030630019_Matematika B_EMT4Plot3D-068.png}
\eulersubheading{}
\begin{eulercomment}
\begin{eulercomment}
\eulerheading{Permukaan Benda Putar}
\begin{eulerprompt}
>plot2d("(x^2+y^2-1)^3-x^2*y^3",r=1.3, ...
>style="#",color=blue,<outline, ...
>level=[-2;0],n=100):
\end{eulerprompt}
\eulerimg{17}{images/Rofi'ah 'Aisy Zukhruf_23030630019_Matematika B_EMT4Plot3D-069.png}
\begin{eulerprompt}
>ekspresi &= (x^2+y^2-1)^3-x^2*y^3; $ekspresi
\end{eulerprompt}
\begin{eulerformula}
\[
\left(y^2+x^2-1\right)^3-x^2\,y^3
\]
\end{eulerformula}
\begin{eulercomment}
Kami ingin memutar kurva jantung di sekitar sumbu y. Inilah ungkapan
yang mendefinisikan hati:

\end{eulercomment}
\begin{eulerformula}
\[
f(x,y)=(x^2+y^2-1)^3-x^2.y^3.
\]
\end{eulerformula}
\begin{eulercomment}
Selanjutnya kita atur

\end{eulercomment}
\begin{eulerformula}
\[
x=r.cos(a),\quad y=r.sin(a).
\]
\end{eulerformula}
\begin{eulerprompt}
>function fr(r,a) &= ekspresi with [x=r*cos(a),y=r*sin(a)] | trigreduce; $fr(r,a)
\end{eulerprompt}
\begin{eulerformula}
\[
\left(r^2-1\right)^3+\frac{\left(\sin \left(5\,a\right)-\sin \left(  3\,a\right)-2\,\sin a\right)\,r^5}{16}
\]
\end{eulerformula}
\begin{eulercomment}
Hal ini memungkinkan untuk menentukan fungsi numerik, yang
menyelesaikan r, jika a diberikan. Dengan fungsi itu kita dapat
memplot jantung yang berubah sebagai permukaan parametrik.
\end{eulercomment}
\begin{eulerprompt}
>function map f(a) := bisect("fr",0,2;a); ...
>t=linspace(-pi/2,pi/2,100); r=f(t);  ...
>s=linspace(pi,2pi,100)'; ...
>plot3d(r*cos(t)*sin(s),r*cos(t)*cos(s),r*sin(t), ...
>>hue,<frame,color=red,zoom=4,amb=0,max=0.7,grid=12,height=50°):
\end{eulerprompt}
\eulerimg{17}{images/Rofi'ah 'Aisy Zukhruf_23030630019_Matematika B_EMT4Plot3D-072.png}
\begin{eulercomment}
Berikut ini adalah plot 3D dari gambar di atas yang diputar di sekitar
sumbu z. Kami mendefinisikan fungsi, yang mendeskripsikan objek.
\end{eulercomment}
\begin{eulerprompt}
>function f(x,y,z) ...
\end{eulerprompt}
\begin{eulerudf}
  r=x^2+y^2;
  return (r+z^2-1)^3-r*z^3;
   endfunction
\end{eulerudf}
\begin{eulerprompt}
>plot3d("f(x,y,z)", ...
>xmin=0,xmax=1.2,ymin=-1.2,ymax=1.2,zmin=-1.2,zmax=1.4, ...
>implicit=1,angle=-30°,zoom=2.5,n=[10,60,60],>anaglyph):
\end{eulerprompt}
\eulerimg{17}{images/Rofi'ah 'Aisy Zukhruf_23030630019_Matematika B_EMT4Plot3D-073.png}
\eulersubheading{}
\begin{eulercomment}
\end{eulercomment}
\eulersubheading{Latihan Soal}
\begin{eulercomment}
1.  Gambarlah permukaan benda putar dari kurva\\
\end{eulercomment}
\begin{eulerformula}
\[
f(x,y)=(x^2+y^2-1)^3-x^2y^3 
\]
\end{eulerformula}
\begin{eulercomment}
dengan memutar kurva ini di sekitar sumbu y. Gunakan parameter r dan a
untuk mendefinisikan permukaan.

JAWAB :
\end{eulercomment}
\begin{eulerprompt}
>ekspresi &= (x^2 + y^2 - 1)^3 - x^2 * y^3;
>function fr(r, a) &= ekspresi with [x = r * cos(a), y = r * sin(a)] | trigreduce;
>function map f(a) := bisect("fr", 0, 2; a); ...
>t = linspace(-pi/2, pi/2, 100); r = f(t); ...
>s = linspace(pi, 2pi, 100)'; ...
>plot3d(r * cos(t) * sin(s), r * cos(t) * cos(s), r * sin(t), ...
> hue, <frame, color=green, zoom=4, amb=0, max=0.7, grid=12, height=50°):
\end{eulerprompt}
\eulerimg{17}{images/Rofi'ah 'Aisy Zukhruf_23030630019_Matematika B_EMT4Plot3D-075.png}
\begin{eulercomment}
2. Buatlah plot 3D dari objek yang dihasilkan dengan memutar kurva
jantung di sekitar sumbu z. Definisikan fungsi\\
\end{eulercomment}
\begin{eulerformula}
\[
f(x,y,z) sebagai (x^2+y^2+z^2-1)^3-r*z^3.
\]
\end{eulerformula}
\begin{eulercomment}
JAWAB :
\end{eulercomment}
\begin{eulerprompt}
>function f(x, y, z) ...
\end{eulerprompt}
\begin{eulerudf}
  r = x^2 + y^2;
  return (r + z^2 - 1)^3 - r * z^3;
  endfunction
\end{eulerudf}
\begin{eulerprompt}
>plot3d("f(x, y, z)", ...
>xmin=0, xmax=1.2, ymin=-1.2, ymax=1.2, zmin=-1.2, zmax=1.4, ...
>implicit=1, angle=-30°, zoom=2.5, n=[10, 60, 60], >anaglyph):
\end{eulerprompt}
\eulerimg{17}{images/Rofi'ah 'Aisy Zukhruf_23030630019_Matematika B_EMT4Plot3D-077.png}
\eulersubheading{}
\begin{eulercomment}
\begin{eulercomment}
\eulerheading{Special 3D Plots}
\begin{eulercomment}
Fungsi plot3d bagus untuk dimiliki, tetapi tidak memenuhi semua
kebutuhan. Selain rutinitas yang lebih mendasar, Anda bisa mendapatkan
plot berbingkai dari objek apa pun yang Anda suka.

Meskipun Euler bukan program 3D, Euler dapat menggabungkan beberapa
objek dasar. Kami mencoba untuk memvisualisasikan paraboloid dan garis
singgung-nya.
\end{eulercomment}
\begin{eulerprompt}
>function myplot ...
\end{eulerprompt}
\begin{eulerudf}
    y=0:0.01:1; x=(0.1:0.01:1)';
    plot3d(x,y,0.2*(x-0.1)/2,<scale,<frame,>hue, ..
      hues=0.5,>contour,color=orange);
    h=holding(1);
    plot3d(x,y,(x^2+y^2)/2,<scale,<frame,>contour,>hue);
    holding(h);
  endfunction
\end{eulerudf}
\begin{eulercomment}
Sekarang framedplot () menyediakan bingkai, dan menyetel tampilan.
\end{eulercomment}
\begin{eulerprompt}
>framedplot("myplot",[0.1,1,0,1,0,1],angle=-45°, ...
>  center=[0,0,-0.7],zoom=6):
\end{eulerprompt}
\eulerimg{17}{images/Rofi'ah 'Aisy Zukhruf_23030630019_Matematika B_EMT4Plot3D-078.png}
\begin{eulercomment}
Dengan cara yang sama, Anda dapat memplot bidang kontur secara manual.
Perhatikan bahwa plot3d () menyetel jendela ke fullwindow () secara
default, tetapi plotcontourplane () mengasumsikannya.
\end{eulercomment}
\begin{eulerprompt}
>x=-1:0.02:1.1; y=x'; z=x^2-y^4;
>function myplot (x,y,z) ...
\end{eulerprompt}
\begin{eulerudf}
    zoom(2);
    wi=fullwindow();
    plotcontourplane(x,y,z,level="auto",<scale);
    plot3d(x,y,z,>hue,<scale,>add,color=white,level="thin");
    window(wi);
    reset();
  endfunction
\end{eulerudf}
\begin{eulerprompt}
>myplot(x,y,z):
\end{eulerprompt}
\eulerimg{27}{images/Rofi'ah 'Aisy Zukhruf_23030630019_Matematika B_EMT4Plot3D-079.png}
\eulersubheading{}
\eulersubheading{Latihan Soal}
\begin{eulercomment}
1. Buatlah plot 3D dari fungsi\\
\end{eulercomment}
\begin{eulerformula}
\[
z=x^2+y^2 (paraboloid)
\]
\end{eulerformula}
\begin{eulercomment}
dan tambahkan garis singgung pada titik (1,1) dengan persamaan garis
singgung z=2(x-1)+2(y-1)+2.

JAWAB :
\end{eulercomment}
\begin{eulerprompt}
>function myplot ...
\end{eulerprompt}
\begin{eulerudf}
  y = 0:0.01:1; 
   x = (0.1:0.01:1)'; ...
  endfunction
\end{eulerudf}
\begin{eulerprompt}
>plot3d(x, y, 2*(x - 1) + 2*(y - 1) + 2, <scale, <frame, >hue, hues=0.5, >contour, color=orange):
\end{eulerprompt}
\eulerimg{27}{images/Rofi'ah 'Aisy Zukhruf_23030630019_Matematika B_EMT4Plot3D-081.png}
\begin{eulerprompt}
>h = holding(1);
>plot3d(x, y, x^2 + y^2, <scale, <frame, >contour, >hue):
\end{eulerprompt}
\eulerimg{27}{images/Rofi'ah 'Aisy Zukhruf_23030630019_Matematika B_EMT4Plot3D-082.png}
\begin{eulerprompt}
>holding(h);
>framedplot("myplot", [0.1, 1, 0, 1, 0, 2], angle=-45°, center=[0, 0, -0.7], zoom=6):
\end{eulerprompt}
\eulerimg{27}{images/Rofi'ah 'Aisy Zukhruf_23030630019_Matematika B_EMT4Plot3D-083.png}
\begin{eulercomment}
2. Buatlah plot bidang kontur dari fungsi\\
\end{eulercomment}
\begin{eulerformula}
\[
z=x^2-y^2
\]
\end{eulerformula}
\begin{eulercomment}
dalam rentang -1\textless{}=x\textless{}=1 dan -1\textless{}=y\textless{}=1. Tambahkan juga plot permukaan
pada bidang kontur tersebut.

JAWAB :
\end{eulercomment}
\begin{eulerprompt}
>x = -1:0.02:1; y = x';
>z = x^2 - y^2;
>function myplot (x, y, z) ...
\end{eulerprompt}
\begin{eulerudf}
  zoom(2);
   wi = fullwindow();
   plotcontourplane(x, y, z, level="auto", <scale);
  plot3d(x, y, z, >hue, <scale, >add, color=white, level="thin");
  window(wi);
  reset();
  endfunction
\end{eulerudf}
\begin{eulerprompt}
>myplot(x, y, z):
\end{eulerprompt}
\eulerimg{27}{images/Rofi'ah 'Aisy Zukhruf_23030630019_Matematika B_EMT4Plot3D-085.png}
\eulersubheading{}
\begin{eulercomment}
\begin{eulercomment}
\eulerheading{Animasi}
\begin{eulercomment}
Salah satu fungsi yang memanfaatkan teknik ini adalah memutar. Itu
dapat mengubah sudut pandang dan menggambar ulang plot 3D. Fungsi
tersebut memanggil addpage () untuk setiap plot baru. Akhirnya itu
menjiwai plot.

Harap pelajari sumber rotasi untuk melihat lebih detail.
\end{eulercomment}
\begin{eulerprompt}
>function testplot () := plot3d("x^2+y^3"); ...
>rotate("testplot"); testplot():
\end{eulerprompt}
\eulerimg{27}{images/Rofi'ah 'Aisy Zukhruf_23030630019_Matematika B_EMT4Plot3D-086.png}
\eulersubheading{}
\begin{eulercomment}
\end{eulercomment}
\eulersubheading{Latihan Soal}
\begin{eulercomment}
1. Buatlah animasi 3D dari fungsi\\
\end{eulercomment}
\begin{eulerformula}
\[
z=sin(x^2+y^2) 
\]
\end{eulerformula}
\begin{eulercomment}
yang berputar di sekitar sumbu z. Tampilkan rotasi ini dengan
menggunakan fungsi rotate.

JAWAB :
\end{eulercomment}
\begin{eulerprompt}
>function testplot() := plot3d("sin(x^2 + y^2)", xmin=-2*pi, xmax=2*pi, ymin=-2*pi, ymax=2*pi, zmin=-1, zmax=1);
>rotate("testplot");
>testplot():
\end{eulerprompt}
\eulerimg{27}{images/Rofi'ah 'Aisy Zukhruf_23030630019_Matematika B_EMT4Plot3D-088.png}
\begin{eulercomment}
2. Buatlah animasi rotasi untuk permukaan\\
\end{eulercomment}
\begin{eulerformula}
\[
z=x^2-y^2 
\]
\end{eulerformula}
\begin{eulercomment}
dengan batas -2\textless{}=x,y\textless{}=2. Tampilkan hasilnya dalam animasi.

JAWAB :
\end{eulercomment}
\begin{eulerprompt}
>function testplot() := plot3d("x^2 - y^2", xmin=-2, xmax=2, ymin=-2, ymax=2, zmin=-4, zmax=4);
>rotate("testplot");
>testplot():
\end{eulerprompt}
\eulerimg{27}{images/Rofi'ah 'Aisy Zukhruf_23030630019_Matematika B_EMT4Plot3D-090.png}
\eulersubheading{}
\begin{eulercomment}
\begin{eulercomment}
\eulerheading{Menggambar Povray}
\begin{eulercomment}
Dengan bantuan file Euler povray.e, Euler dapat menghasilkan file
Povray. Hasilnya sangat bagus untuk dilihat.

Anda perlu menginstal Povray (32bit atau 64bit)dari\\
http://www.povray.org/,\\
dan meletakkan sub-direktori "bin" Povray ke dalam jalur lingkungan,
atau menyetel variabel "defaultpovray" dengan jalur lengkap yang
mengarah ke "pvengine.exe".

Antarmuka Povray dari Euler menghasilkan file Povray di direktori home
pengguna, dan memanggil Povray untuk mengurai file-file ini. Nama file
default adalah current.pov, dan direktori default adalah eulerhome (),
biasanya c: \textbackslash{} Users \textbackslash{} Username \textbackslash{} Euler. Povray menghasilkan file PNG,
yang dapat dimuat oleh Euler ke dalam notebook. Untuk membersihkan
file-file ini, gunakan povclear ().

Fungsi pov3d memiliki semangat yang sama dengan plot3d. Ini dapat
menghasilkan grafik fungsi f (x, y), atau permukaan dengan koordinat
X, Y, Z dalam matriks, termasuk garis level opsional. Fungsi ini
memulai raytracer secara otomatis, dan memuat pemandangan ke dalam
notebook Euler.

Selain pov3d (), ada banyak fungsi yang menghasilkan objek Povray.
Fungsi-fungsi ini mengembalikan string, berisi kode Povray untuk
objek. Untuk menggunakan fungsi ini, mulai file Povray dengan povstart
(). Kemudian gunakan writeln (...) untuk menulis objek ke file adegan.
Terakhir, akhiri file dengan povend (). Secara default, raytracer akan
mulai, dan PNG akan dimasukkan ke dalam notebook Euler.

Fungsi objek memiliki parameter yang disebut "look", yang membutuhkan
string dengan kode Povray untuk tekstur dan penyelesaian objek. Fungsi
povlook () dapat digunakan untuk menghasilkan string ini. Ini memiliki
parameter untuk warna, transparansi, Phong Shading dll.

Perhatikan bahwa alam semesta Povray memiliki sistem koordinat lain.
Antarmuka ini menerjemahkan semua koordinat ke sistem Povray. Jadi
Anda dapat terus berpikir dalam sistem koordinat Euler dengan z
menunjuk ke atas secara vertikal, sumbu a nd x, y, z di tangan kanan.\\
Anda perlu memuat file povray.
\end{eulercomment}
\begin{eulerprompt}
>load povray;
\end{eulerprompt}
\begin{eulercomment}
Pastikan, direktori bin Povray ada di jalurnya. Jika tidak, edit
variabel berikut sehingga berisi path ke povray yang dapat dieksekusi.
\end{eulercomment}
\begin{eulerprompt}
>defaultpovray="C:\(\backslash\)Program Files\(\backslash\)POV-Ray\(\backslash\)v3.7\(\backslash\)bin\(\backslash\)pvengine.exe"
\end{eulerprompt}
\begin{euleroutput}
  C:\(\backslash\)Program Files\(\backslash\)POV-Ray\(\backslash\)v3.7\(\backslash\)bin\(\backslash\)pvengine.exe
\end{euleroutput}
\begin{eulercomment}
Untuk pertamakali, kami memplot fungsi sederhana. Perintah berikut
menghasilkan file povray di direktori pengguna Anda, dan menjalankan
Povray untuk menelusuri file ini.

Jika Anda memulai perintah berikut, GUI Povray akan terbuka,
menjalankan file, dan menutup secara otomatis. Karena alasan keamanan,
Anda akan ditanya, apakah Anda ingin mengizinkan file exe dijalankan.
Anda dapat menekan batal untuk menghentikan pertanyaan lebih lanjut.
Anda mungkin harus menekan OK di jendela Povray untuk mengetahui
dialog start-up Povray.
\end{eulercomment}
\begin{eulerprompt}
>plot3d("x^2+y^2",zoom=2):
\end{eulerprompt}
\eulerimg{27}{images/Rofi'ah 'Aisy Zukhruf_23030630019_Matematika B_EMT4Plot3D-091.png}
\begin{eulerprompt}
>pov3d("x^2+y^2",zoom=3);
\end{eulerprompt}
\eulerimg{28}{images/Rofi'ah 'Aisy Zukhruf_23030630019_Matematika B_EMT4Plot3D-092.png}
\begin{eulercomment}
Kita bisa membuat fungsinya transparan dan menambahkan hasil akhir
lainnya. Kita juga bisa menambahkan garis level ke plot fungsi.
\end{eulercomment}
\begin{eulerprompt}
>pov3d("x^2+y^3",axiscolor=red,angle=20°, ...
>  look=povlook(blue,0.2),level=-1:0.5:1,zoom=3.8);
\end{eulerprompt}
\eulerimg{28}{images/Rofi'ah 'Aisy Zukhruf_23030630019_Matematika B_EMT4Plot3D-093.png}
\begin{eulercomment}
Terkadang perlu untuk mencegah penskalaan fungsi, dan menskalakan
fungsi dengan tangan.

Kami memplot himpunan titik di bidang kompleks, di mana hasil kali
jarak ke 1 dan -1 sama dengan 1.
\end{eulercomment}
\begin{eulerprompt}
>pov3d("((x-1)^2+y^2)*((x+1)^2+y^2)/40",r=1.5, ...
>  angle=-120°,level=1/40,dlevel=0.005,light=[-1,1,1],height=45°,n=50, ...
>  <fscale,zoom=3.8);
\end{eulerprompt}
\eulerimg{28}{images/Rofi'ah 'Aisy Zukhruf_23030630019_Matematika B_EMT4Plot3D-094.png}
\eulersubheading{}
\begin{eulercomment}
\end{eulercomment}
\eulersubheading{Latihan Soal}
\begin{eulercomment}
1.  Plotlah fungsi\\
\end{eulercomment}
\begin{eulerformula}
\[
z=sin(x)*cos(y)
\]
\end{eulerformula}
\begin{eulercomment}
dalam ruang 3D. Tambahkan garis level pada plot dengan rentang level
dari -1 hingga 1

JAWAB :
\end{eulercomment}
\begin{eulerprompt}
>f(x, y) := sin(x) * cos(y);
\end{eulerprompt}
\begin{euleroutput}
  Variable or function x not found.
  Error in:
  f(x, y) := sin(x) * cos(y); ...
     ^
\end{euleroutput}
\begin{eulerprompt}
>pov3d("sin(x)*cos(y)", axiscolor=blue, angle=30°, ...
>look=povlook(red, 0.3), level=-1:0.5:1, zoom=3):
\end{eulerprompt}
\eulerimg{28}{images/Rofi'ah 'Aisy Zukhruf_23030630019_Matematika B_EMT4Plot3D-096.png}
\eulerimg{27}{images/Rofi'ah 'Aisy Zukhruf_23030630019_Matematika B_EMT4Plot3D-097.png}
\begin{eulercomment}
2. Buatlah permukaan dari fungsi\\
\end{eulercomment}
\begin{eulerformula}
\[
z=x^2+y^2 
\]
\end{eulerformula}
\begin{eulercomment}
dan tambahkan vektor normal pada setiap titik permukaan. Gunakan
rentang x dan y dari -2 hingga 2.

JAWAB :
\end{eulercomment}
\begin{eulerprompt}
>Z &= x^2 + y^2;
>dx &= diff([x, y, Z], x);
>dy &= diff([x, y, Z], y);
>N &= crossproduct(dx, dy);
>NX &= N[1]; NY &= N[2]; NZ &= N[3];
>x = -2:0.5:2;
>y = x';
>pov3d(x, y, Z(x, y), angle=15°, ...
>xv=NX(x, y), yv=NY(x, y), zv=NZ(x, y), <shadow):
\end{eulerprompt}
\eulerimg{28}{images/Rofi'ah 'Aisy Zukhruf_23030630019_Matematika B_EMT4Plot3D-099.png}
\eulerimg{27}{images/Rofi'ah 'Aisy Zukhruf_23030630019_Matematika B_EMT4Plot3D-100.png}
\eulersubheading{}
\begin{eulercomment}
\begin{eulercomment}
\eulerheading{Merencanakan dengan Koordinat}
\begin{eulercomment}
Alih-alih fungsi, kita bisa memplot dengan koordinat. Seperti pada
plot3d, kita membutuhkan tiga matriks untuk mendefinisikan objeknya.

Dalam contoh ini, kita memutar fungsi di sekitar sumbu z.
\end{eulercomment}
\begin{eulerprompt}
>function f(x) := x^3-x+1; ...
>x=-1:0.01:1; t=linspace(0,2pi,8)'; ...
>Z=x; X=cos(t)*f(x); Y=sin(t)*f(x); ...
>pov3d(X,Y,Z,angle=40°,height=20°,axis=0,zoom=4,light=[10,-5,5]);
\end{eulerprompt}
\eulerimg{28}{images/Rofi'ah 'Aisy Zukhruf_23030630019_Matematika B_EMT4Plot3D-101.png}
\begin{eulercomment}
Dalam contoh berikut, kami memplot gelombang teredam. Kami
menghasilkan gelombang dengan bahasa matriks Euler.

Kami juga menunjukkan, bagaimana objek tambahan dapat ditambahkan ke
adegan pov3d. Untuk pembuatan objek, lihat contoh berikut. Perhatikan
bahwa plot3d menskalakan plot, sehingga sesuai dengan kubus satuan
\end{eulercomment}
\begin{eulerprompt}
>r=linspace(0,1,80); phi=linspace(0,2pi,80)'; ...
>x=r*cos(phi); y=r*sin(phi); z=exp(-5*r)*cos(8*pi*r)/3;  ...
>pov3d(x,y,z,zoom=5,axis=0,add=povsphere([0,0,0.5],0.1,povlook(green)), ...
>  w=500,h=300);
\end{eulerprompt}
\eulerimg{16}{images/Rofi'ah 'Aisy Zukhruf_23030630019_Matematika B_EMT4Plot3D-102.png}
\begin{eulercomment}
Dengan metode naungan lanjutan Povray, sangat sedikit titik yang dapat
menghasilkan permukaan yang sangat halus. Hanya di perbatasan dan
dalam bayangan, triknya mungkin menjadi jelas.

Untuk ini, kita perlu menambahkan vektor normal di setiap titik
matriks.
\end{eulercomment}
\begin{eulerprompt}
>Z &= x^2*y^3
\end{eulerprompt}
\begin{euleroutput}
  
                                   2  3
                                  x  y
  
\end{euleroutput}
\begin{eulercomment}
Persamaan permukaannya adalah [x, y, Z]. Kami menghitung dua turunan
menjadi x dan y dari ini dan mengambil produk silang sebagai normal.
\end{eulercomment}
\begin{eulerprompt}
>dx &= diff([x,y,Z],x); dy &= diff([x,y,Z],y);
\end{eulerprompt}
\begin{eulercomment}
Kami mendefinisikan normal sebagai produk silang dari turunan ini, dan
mendefinisikan fungsi koordinat.
\end{eulercomment}
\begin{eulerprompt}
>N &= crossproduct(dx,dy); NX &= N[1]; NY &= N[2]; NZ &= N[3]; N,
\end{eulerprompt}
\begin{euleroutput}
  
                                 3       2  2
                         [- 2 x y , - 3 x  y , 1]
  
\end{euleroutput}
\begin{eulercomment}
We use only 25 points.
\end{eulercomment}
\begin{eulerprompt}
>x=-1:0.5:1; y=x';
>pov3d(x,y,Z(x,y),angle=10°, ...
>  xv=NX(x,y),yv=NY(x,y),zv=NZ(x,y),<shadow);
\end{eulerprompt}
\eulerimg{28}{images/Rofi'ah 'Aisy Zukhruf_23030630019_Matematika B_EMT4Plot3D-103.png}
\begin{eulercomment}
Berikut ini adalah simpul Trefoil yang dilakukan oleh A. Busser di
Povray. Ada versi perbaikannya dalam contoh.


See: Examples\textbackslash{}Trefoil Knot \textbar{} Trefoil Knot

Untuk tampilan yang bagus dengan tidak terlalu banyak titik, kami
menambahkan vektor normal di sini. Kami menggunakan Maxima untuk
menghitung normal bagi kami. Pertama, tiga fungsi koordinat sebagai
ekspresi simbolik.
\end{eulercomment}
\begin{eulerprompt}
>X &= ((4+sin(3*y))+cos(x))*cos(2*y); ...
>Y &= ((4+sin(3*y))+cos(x))*sin(2*y); ...
>Z &= sin(x)+2*cos(3*y);
\end{eulerprompt}
\begin{eulercomment}
Kemudian dua vektor turunannya menjadi x dan y.
\end{eulercomment}
\begin{eulerprompt}
>dx &= diff([X,Y,Z],x); dy &= diff([X,Y,Z],y);
\end{eulerprompt}
\begin{eulercomment}
Sekarang normal, yang merupakan produk persilangan dari dua
turunannya.
\end{eulercomment}
\begin{eulerprompt}
>dn &= crossproduct(dx,dy);
\end{eulerprompt}
\begin{eulercomment}
Kami sekarang mengevaluasi semua ini secara numerik.
\end{eulercomment}
\begin{eulerprompt}
>x:=linspace(-%pi,%pi,40); y:=linspace(-%pi,%pi,100)';
\end{eulerprompt}
\begin{eulercomment}
Vektor normal adalah evaluasi dari ekspresi simbolik dn [i] untuk i =
1,2,3. Sintaks untuk ini adalah \& "expresi" (parameter). Ini adalah
alternatif metode pada contoh sebelumnya, di mana kita mendefinisikan
ekspresi simbolik NX, NY, NZ terlebih dahulu.
\end{eulercomment}
\begin{eulerprompt}
>pov3d(X(x,y),Y(x,y),Z(x,y),axis=0,zoom=5,w=450,h=350, ...
>  <shadow,look=povlook(gray), ...
>  xv=&"dn[1]"(x,y), yv=&"dn[2]"(x,y), zv=&"dn[3]"(x,y));
\end{eulerprompt}
\eulerimg{21}{images/Rofi'ah 'Aisy Zukhruf_23030630019_Matematika B_EMT4Plot3D-104.png}
\begin{eulercomment}
Kami juga dapat membuat grid dalam 3D.
\end{eulercomment}
\begin{eulerprompt}
>povstart(zoom=4); ...
>x=-1:0.5:1; r=1-(x+1)^2/6; ...
>t=(0°:30°:360°)'; y=r*cos(t); z=r*sin(t); ...
>writeln(povgrid(x,y,z,d=0.02,dballs=0.05)); ...
>povend();
\end{eulerprompt}
\eulerimg{28}{images/Rofi'ah 'Aisy Zukhruf_23030630019_Matematika B_EMT4Plot3D-105.png}
\begin{eulercomment}
Dengan povgrid (), kurva dimungkinkan.
\end{eulercomment}
\begin{eulerprompt}
>povstart(center=[0,0,1],zoom=3.6); ...
>t=linspace(0,2,1000); r=exp(-t); ...
>x=cos(2*pi*10*t)*r; y=sin(2*pi*10*t)*r; z=t; ...
>writeln(povgrid(x,y,z,povlook(red))); ...
>writeAxis(0,2,axis=3); ...
>povend();
\end{eulerprompt}
\eulerimg{28}{images/Rofi'ah 'Aisy Zukhruf_23030630019_Matematika B_EMT4Plot3D-106.png}
\eulersubheading{}
\begin{eulercomment}
\end{eulercomment}
\eulersubheading{Latihan Soal}
\begin{eulercomment}
1. Plotlah fungsi\\
\end{eulercomment}
\begin{eulerformula}
\[
f(x)=x^3-x+1
\]
\end{eulerformula}
\begin{eulercomment}
yang diputar di sekitar sumbu z. Tampilkan hasilnya dengan sudut
pandang tertentu.

JAWAB :
\end{eulercomment}
\begin{eulerprompt}
>function f(x) := x^3 - x + 1;
>x = -1:0.01:1;
>t = linspace(0, 2*pi, 8)';
>Z = x; 
>X = cos(t) * f(x);
>Y = sin(t) * f(x);
>pov3d(X, Y, Z, angle=40°, height=20°, axis=0, zoom=4, light=[10, -5, 5]);
\end{eulerprompt}
\eulerimg{28}{images/Rofi'ah 'Aisy Zukhruf_23030630019_Matematika B_EMT4Plot3D-108.png}
\begin{eulercomment}
2. Plotlah fungsi\\
\end{eulercomment}
\begin{eulerformula}
\[
z=x^2y^3
\]
\end{eulerformula}
\begin{eulercomment}
di dalam ruang 3D. Hitunglah normal di setiap titik permukaan dan
tampilkan dengan vektor normal pada plot.

JAWAB :
\end{eulercomment}
\begin{eulerprompt}
>Z &= x^2 * y^3;
>dx &= diff([x, y, Z], x);
>dy &= diff([x, y, Z], y);
>N &= crossproduct(dx, dy);
>NX &= N[1]; NY &= N[2]; NZ &= N[3]; N;
>x = -1:0.5:1; y = x';
>pov3d(x, y, Z(x, y), angle=10°, ...
>xv=NX(x, y), yv=NY(x, y), zv=NZ(x, y), <shadow):
\end{eulerprompt}
\eulerimg{28}{images/Rofi'ah 'Aisy Zukhruf_23030630019_Matematika B_EMT4Plot3D-110.png}
\eulerimg{27}{images/Rofi'ah 'Aisy Zukhruf_23030630019_Matematika B_EMT4Plot3D-111.png}
\eulersubheading{}
\begin{eulercomment}
\begin{eulercomment}
\eulerheading{Objek Povray}
\begin{eulercomment}
Di atas, kami menggunakan pov3d untuk memplot permukaan. Antarmuka
povray di Euler juga dapat menghasilkan objek Povray. Objek-objek ini
disimpan sebagai string di Euler, dan perlu ditulis ke file Povray.

Kami memulai output dengan povstart ().
\end{eulercomment}
\begin{eulerprompt}
>povstart(zoom=4);
\end{eulerprompt}
\begin{eulercomment}
Pertama kita tentukan tiga silinder, dan simpan dalam string di Euler.

Fungsi povx () dll. Hanya mengembalikan vektor [1,0,0], yang bisa
digunakan sebagai gantinya.
\end{eulercomment}
\begin{eulerprompt}
>c1=povcylinder(-povx,povx,1,povlook(red)); ...
>c2=povcylinder(-povy,povy,1,povlook(green)); ...
>c3=povcylinder(-povz,povz,1,povlook(blue)); ...
\end{eulerprompt}
\begin{eulercomment}
String berisi kode Povray, yang tidak perlu kita pahami pada saat itu.
\end{eulercomment}
\begin{eulerprompt}
>c2
\end{eulerprompt}
\begin{euleroutput}
  cylinder \{ <0,0,-1>, <0,0,1>, 1
   texture \{ pigment \{ color rgb <0.0627451,0.564706,0.0627451> \}  \} 
   finish \{ ambient 0.2 \} 
   \}
\end{euleroutput}
\begin{eulercomment}
Seperti yang Anda lihat, kami menambahkan tekstur ke objek dalam tiga
warna berbeda.

Itu dilakukan oleh povlook (), yang mengembalikan string dengan kode
Povray yang relevan. Kita dapat menggunakan warna Euler default, atau
menentukan warna kita sendiri. Kami juga dapat menambahkan
transparansi, atau mengubah cahaya sekitar.
\end{eulercomment}
\begin{eulerprompt}
>povlook(rgb(0.1,0.2,0.3),0.1,0.5)
\end{eulerprompt}
\begin{euleroutput}
   texture \{ pigment \{ color rgbf <0.101961,0.2,0.301961,0.1> \}  \} 
   finish \{ ambient 0.5 \} 
  
\end{euleroutput}
\begin{eulercomment}
Sekarang kita mendefinisikan objek interseksi, dan menulis hasilnya ke
file.
\end{eulercomment}
\begin{eulerprompt}
>writeln(povintersection([c1,c2,c3]));
\end{eulerprompt}
\begin{eulercomment}
Perpotongan tiga silinder sulit untuk divisualisasikan, jika Anda
belum pernah melihatnya sebelumnya.
\end{eulercomment}
\begin{eulerprompt}
>povend;
\end{eulerprompt}
\eulerimg{28}{images/Rofi'ah 'Aisy Zukhruf_23030630019_Matematika B_EMT4Plot3D-112.png}
\begin{eulercomment}
Fungsi berikut menghasilkan fraktal secara rekursif.

Fungsi pertama menunjukkan, bagaimana Euler menangani objek Povray
sederhana. Fungsi povbox () mengembalikan string, berisi koordinat
kotak, tekstur, dan hasil akhir.
\end{eulercomment}
\begin{eulerprompt}
>function onebox(x,y,z,d) := povbox([x,y,z],[x+d,y+d,z+d],povlook());
>function fractal (x,y,z,h,n) ...
\end{eulerprompt}
\begin{eulerudf}
   if n==1 then writeln(onebox(x,y,z,h));
   else
     h=h/3;
     fractal(x,y,z,h,n-1);
     fractal(x+2*h,y,z,h,n-1);
     fractal(x,y+2*h,z,h,n-1);
     fractal(x,y,z+2*h,h,n-1);
     fractal(x+2*h,y+2*h,z,h,n-1);
     fractal(x+2*h,y,z+2*h,h,n-1);
     fractal(x,y+2*h,z+2*h,h,n-1);
     fractal(x+2*h,y+2*h,z+2*h,h,n-1);
     fractal(x+h,y+h,z+h,h,n-1);
   endif;
  endfunction
\end{eulerudf}
\begin{eulerprompt}
>povstart(fade=10,<shadow);
>fractal(-1,-1,-1,2,4);
>povend();
\end{eulerprompt}
\eulerimg{28}{images/Rofi'ah 'Aisy Zukhruf_23030630019_Matematika B_EMT4Plot3D-113.png}
\begin{eulercomment}
Perbedaan memungkinkan pemotongan satu objek dari yang lain. Seperti
persimpangan, ada bagian dari objek CSG Povray.
\end{eulercomment}
\begin{eulerprompt}
>povstart(light=[5,-5,5],fade=10);
\end{eulerprompt}
\begin{eulercomment}
Untuk demonstrasi ini, kami mendefinisikan sebuah objek di Povray,
daripada menggunakan string di Euler. Definisi segera ditulis ke file.

Koordinat kotak -1 berarti [-1, -1, -1].
\end{eulercomment}
\begin{eulerprompt}
>povdefine("mycube",povbox(-1,1));
\end{eulerprompt}
\begin{eulercomment}
Kita bisa menggunakan objek ini di povobject (), yang mengembalikan
string seperti biasa.
\end{eulercomment}
\begin{eulerprompt}
>c1=povobject("mycube",povlook(red));
\end{eulerprompt}
\begin{eulercomment}
Kami menghasilkan kubus kedua, dan memutar serta menskalakannya
sedikit.
\end{eulercomment}
\begin{eulerprompt}
>c2=povobject("mycube",povlook(yellow),translate=[1,1,1], ...
>  rotate=xrotate(10°)+yrotate(10°), scale=1.2);
\end{eulerprompt}
\begin{eulercomment}
Kemudian kita ambil perbedaan kedua objek tersebut.
\end{eulercomment}
\begin{eulerprompt}
>writeln(povdifference(c1,c2));
\end{eulerprompt}
\begin{eulercomment}
Sekarang tambahkan tiga sumbu.
\end{eulercomment}
\begin{eulerprompt}
>writeAxis(-1.2,1.2,axis=1); ...
>writeAxis(-1.2,1.2,axis=2); ...
>writeAxis(-1.2,1.2,axis=4); ...
>povend();
\end{eulerprompt}
\eulerimg{28}{images/Rofi'ah 'Aisy Zukhruf_23030630019_Matematika B_EMT4Plot3D-114.png}
\eulersubheading{}
\begin{eulercomment}
\end{eulercomment}
\eulersubheading{Latihan Soal}
\begin{eulercomment}
1. Buatlah tiga silinder dengan warna merah, hijau, dan biru yang
saling berpotongan. Tampilkan objek-objek ini dan simpan hasilnya ke
dalam file Povray.

JAWAB :
\end{eulercomment}
\begin{eulerprompt}
>povstart(zoom=4);
>c1 = povcylinder(-povx, povx, 1, povlook(red));
>c2 = povcylinder(-povy, povy, 1, povlook(green));
>c3 = povcylinder(-povz, povz, 1, povlook(blue));
>writeln(povintersection([c1, c2, c3]));
>povend();
\end{eulerprompt}
\eulerimg{28}{images/Rofi'ah 'Aisy Zukhruf_23030630019_Matematika B_EMT4Plot3D-115.png}
\begin{eulercomment}
2. Buatlah dua kubus yang saling berpotongan, satu kubus berwarna
merah dan yang lainnya berwarna kuning. Kubus kedua harus sedikit
dipindahkan dan diputar. Tampilkan hasil potongan objek tersebut.

JAWAB :
\end{eulercomment}
\begin{eulerprompt}
>povstart(light=[5,-5,5], fade=10);
>povdefine("mycube", povbox(-1, 1));
>c1 = povobject("mycube", povlook(red));
>c2 = povobject("mycube", povlook(yellow), translate=[1, 1, 1], ...
> rotate=xrotate(10°) + yrotate(10°), scale=1.2);
>writeln(povdifference(c1, c2));
>writeAxis(-1.2, 1.2, axis=1);
>writeAxis(-1.2, 1.2, axis=2);
>writeAxis(-1.2, 1.2, axis=4);
>povend();
\end{eulerprompt}
\eulerimg{28}{images/Rofi'ah 'Aisy Zukhruf_23030630019_Matematika B_EMT4Plot3D-116.png}
\eulersubheading{}
\begin{eulercomment}
\begin{eulercomment}
\eulerheading{Fungsi Implisit}
\begin{eulercomment}
Povray dapat memplot himpunan di mana f (x, y, z) = 0, seperti
parameter implisit di plot3d. Namun, hasilnya terlihat jauh lebih
baik.

Sintaks untuk fungsinya sedikit berbeda. Anda tidak dapat menggunakan
keluaran ekspresi Maxima atau Euler.

\end{eulercomment}
\begin{eulerformula}
\[
((x^2+y^2-c^2)^2+(z^2-1)^2)*((y^2+z^2-c^2)^2+(x^2-1)^2)*((z^2+x^2-c^2)^2+(y^2-1)^2)=d
\]
\end{eulerformula}
\begin{eulerprompt}
>povstart(angle=70°,height=50°,zoom=4);
>c=0.1; d=0.1; ...
>writeln(povsurface("(pow(pow(x,2)+pow(y,2)-pow(c,2),2)+pow(pow(z,2)-1,2))*(pow(pow(y,2)+pow(z,2)-pow(c,2),2)+pow(pow(x,2)-1,2))*(pow(pow(z,2)+pow(x,2)-pow(c,2),2)+pow(pow(y,2)-1,2))-d",povlook(red))); ...
>povend();
\end{eulerprompt}
\begin{euleroutput}
  Error : Povray error!
  
  Error generated by error() command
  
  povray:
      error("Povray error!");
  Try "trace errors" to inspect local variables after errors.
  povend:
      povray(file,w,h,aspect,exit); 
\end{euleroutput}
\begin{eulerprompt}
>povstart(angle=25°,height=10°);
>writeln(povsurface("pow(x,2)+pow(y,2)*pow(z,2)-1",povlook(blue),povbox(-2,2,"")));
>povend();
\end{eulerprompt}
\eulerimg{28}{images/Rofi'ah 'Aisy Zukhruf_23030630019_Matematika B_EMT4Plot3D-118.png}
\begin{eulerprompt}
>povstart(angle=70°,height=50°,zoom=4);
\end{eulerprompt}
\begin{eulercomment}
Buat permukaan implisit. Perhatikan sintaks yang berbeda dalam
ekspresi tersebut.
\end{eulercomment}
\begin{eulerprompt}
>writeln(povsurface("pow(x,2)*y-pow(y,3)-pow(z,2)",povlook(green))); ...
>writeAxes(); ...
>povend();
\end{eulerprompt}
\eulerimg{28}{images/Rofi'ah 'Aisy Zukhruf_23030630019_Matematika B_EMT4Plot3D-119.png}
\eulersubheading{}
\begin{eulercomment}
\end{eulercomment}
\eulersubheading{Latihan Soal}
\begin{eulercomment}
1. Buatlah permukaan implisit untuk persamaan\\
\end{eulercomment}
\begin{eulerformula}
\[
(x^2+y^2+z^2-1)*(x^2+y^2-z^2)=0. 
\]
\end{eulerformula}
\begin{eulercomment}
Tampilkan permukaan ini dengan warna merah.

JAWAB :
\end{eulercomment}
\begin{eulerprompt}
>povstart(angle=60°, height=50°, zoom=5);
>writeln(povsurface("(pow(x,2) + pow(y,2) + pow(z,2) - 1) * (pow(x,2) + pow(y,2) - pow(z,2))", povlook(red)));
>writeAxes(); 
>writeAxes():
\end{eulerprompt}
\eulerimg{27}{images/Rofi'ah 'Aisy Zukhruf_23030630019_Matematika B_EMT4Plot3D-121.png}
\begin{eulercomment}
2. Buatlah permukaan implisit untuk persamaan\\
\end{eulercomment}
\begin{eulerformula}
\[
z-x^2-y^2=0 
\]
\end{eulerformula}
\begin{eulercomment}
yang merepresentasikan paraboloid. Tampilkan permukaan ini dengan
warna biru dan tambahkan sumbu untuk referensi.

JAWAB :
\end{eulercomment}
\begin{eulerprompt}
>povstart(angle=30°, height=20°, zoom=5);
>writeln(povsurface("z - pow(x,2) - pow(y,2)", povlook(blue)));
>writeAxes();
>povend();
\end{eulerprompt}
\eulerimg{28}{images/Rofi'ah 'Aisy Zukhruf_23030630019_Matematika B_EMT4Plot3D-123.png}
\eulersubheading{}
\begin{eulercomment}
\begin{eulercomment}
\eulerheading{Objek Jaring}
\begin{eulercomment}
Dalam contoh ini, kami menunjukkan cara membuat objek mesh, dan
menggambarnya dengan informasi tambahan.

Kami ingin memaksimalkan xy di bawah kondisi x + y = 1 dan
mendemonstrasikan sentuhan tangensial dari garis level.
\end{eulercomment}
\begin{eulerprompt}
>povstart(angle=-10°,center=[0.5,0.5,0.5],zoom=7);
\end{eulerprompt}
\begin{eulercomment}
Kami tidak dapat menyimpan objek dalam string seperti sebelumnya,
karena terlalu besar. Jadi kami mendefinisikan objek dalam file Povray
menggunakan #declare. Fungsi povtriangle () melakukan ini secara
otomatis. Ia dapat menerima vektor normal seperti pov3d ().

Yang berikut ini mendefinisikan objek mesh, dan langsung menulisnya ke
dalam file.
\end{eulercomment}
\begin{eulerprompt}
>x=0:0.02:1; y=x'; z=x*y; vx=-y; vy=-x; vz=1;
>mesh=povtriangles(x,y,z,"",vx,vy,vz);
\end{eulerprompt}
\begin{eulercomment}
Sekarang kami mendefinisikan dua disk, yang akan berpotongan dengan
permukaan.
\end{eulercomment}
\begin{eulerprompt}
>cl=povdisc([0.5,0.5,0],[1,1,0],2); ...
>ll=povdisc([0,0,1/4],[0,0,1],2);
\end{eulerprompt}
\begin{eulercomment}
Tulis permukaan dikurangi dua cakram.
\end{eulercomment}
\begin{eulerprompt}
>writeln(povdifference(mesh,povunion([cl,ll]),povlook(green)));
\end{eulerprompt}
\begin{eulercomment}
Tuliskan dua persimpangan tersebut.
\end{eulercomment}
\begin{eulerprompt}
>writeln(povintersection([mesh,cl],povlook(red))); ...
>writeln(povintersection([mesh,ll],povlook(gray)));
\end{eulerprompt}
\begin{eulercomment}
Tulis titik maksimal.
\end{eulercomment}
\begin{eulerprompt}
>writeln(povpoint([1/2,1/2,1/4],povlook(gray),size=2*defaultpointsize));
\end{eulerprompt}
\begin{eulercomment}
Tambahkan sumbu dan selesai.
\end{eulercomment}
\begin{eulerprompt}
>writeAxes(0,1,0,1,0,1,d=0.015); ...
>povend();
\end{eulerprompt}
\eulerimg{28}{images/Rofi'ah 'Aisy Zukhruf_23030630019_Matematika B_EMT4Plot3D-124.png}
\eulersubheading{}
\begin{eulercomment}
\end{eulercomment}
\eulersubheading{Latihan Soal}
\begin{eulercomment}
1. Buatlah objek mesh yang mewakili permukaan\\
\end{eulercomment}
\begin{eulerformula}
\[
z=x^2+y^2
\]
\end{eulerformula}
\begin{eulercomment}
untuk x dan y dalam interval [0, 1]. Tambahkan sebuah disk yang
berpotongan dengan permukaan tersebut pada posisi(0.5,0.5,0) dengan
jari-jari 0.3. Tampilkan hasil pemotongan dalam warna biru.

JAWAB :
\end{eulercomment}
\begin{eulerprompt}
>povstart(angle=-30°, center=[0.5, 0.5, 0.5], zoom=6);
>x = 0:0.02:1; y = x'; 
>z = x.^2 + y.^2; 
>vx = -2*x;
>vy = -2*y;
>vz = ones(size(x)); 
>mesh = povtriangles(x, y, z, "", vx, vy, vz);
>disk = povdisc([0.5, 0.5, 0], [0, 0, 1], 0.3);
>writeln(povdifference(mesh, disk, povlook(blue)));
>writeAxes(0, 1, 0, 1, 0, 1, d=0.015); 
>povend();
\end{eulerprompt}
\eulerimg{28}{images/Rofi'ah 'Aisy Zukhruf_23030630019_Matematika B_EMT4Plot3D-126.png}
\begin{eulercomment}
2. Buatlah objek mesh untuk fungsi\\
\end{eulercomment}
\begin{eulerformula}
\[
z=1-x^2-y^2 
\]
\end{eulerformula}
\begin{eulercomment}
untuk x dan y dalam interval [-1, 1]. Tambahkan dua cakram yang
berpotongan dengan permukaan tersebut: satu di (0,0,0) dengan
jari-jari 0.5 dan yang lainnya di (0,0,0.5) dengan jari-jari 0.2.
Tampilkan hasil pemotongan dengan warna ungu dan tunjukkan titik
maksimum pada permukaan.

JAWAB:
\end{eulercomment}
\begin{eulerprompt}
>povstart(angle=-10°, center=[0, 0, 0.5], zoom=5);
>x = -1:0.02:1; y = x';
>z = 1 - x.^2 - y.^2;
>vx = -2*x;
>vy = -2*y;
>vz = ones(size(x));
>mesh = povtriangles(x, y, z, "", vx, vy, vz);
>disk1 = povdisc([0, 0, 0], [0, 0, 1], 0.5);
>disk2 = povdisc([0, 0, 0.5], [0, 0, 1], 0.2);
>writeln(povdifference(mesh, povunion([disk1, disk2]), povlook(purple)));
\end{eulerprompt}
\begin{euleroutput}
  Variable or function purple not found.
  Error in:
  ... ce(mesh, povunion([disk1, disk2]), povlook(purple))); ...
                                                       ^
\end{euleroutput}
\begin{eulerprompt}
>writeln(povpoint([0, 0, 1], povlook(gray), size=2*defaultpointsize));
>writeAxes(-1, 1, -1, 1, 0, 1, d=0.015); 
>povend();
\end{eulerprompt}
\eulerimg{28}{images/Rofi'ah 'Aisy Zukhruf_23030630019_Matematika B_EMT4Plot3D-128.png}
\eulersubheading{}
\begin{eulercomment}
\begin{eulercomment}
\eulerheading{Anaglyph di Povray}
\begin{eulercomment}
Untuk menghasilkan anaglyph untuk kacamata merah / cyan, Povray harus
dijalankan dua kali dari posisi kamera yang berbeda. Ini menghasilkan
dua file Povray dan dua file PNG, yang dimuat dengan fungsi
loadanaglyph ().

Tentu saja, Anda memerlukan kaca mata merah / cyan untuk melihat
contoh berikut dengan benar.

Fungsi pov3d () memiliki tombol sederhana untuk menghasilkan anaglyph.
\end{eulercomment}
\begin{eulerprompt}
>pov3d("-exp(-x^2-y^2)/2",r=2,height=45°,>anaglyph, ...
>  center=[0,0,0.5],zoom=3.5);
\end{eulerprompt}
\eulerimg{27}{images/Rofi'ah 'Aisy Zukhruf_23030630019_Matematika B_EMT4Plot3D-129.png}
\begin{eulercomment}
Jika Anda membuat adegan dengan objek, Anda perlu memasukkan pembuatan
adegan ke dalam fungsi, dan menjalankannya dua kali dengan nilai yang
berbeda untuk parameter anaglyph.
\end{eulercomment}
\begin{eulerprompt}
>function myscene ...
\end{eulerprompt}
\begin{eulerudf}
    s=povsphere(povc,1);
    cl=povcylinder(-povz,povz,0.5);
    clx=povobject(cl,rotate=xrotate(90°));
    cly=povobject(cl,rotate=yrotate(90°));
    c=povbox([-1,-1,0],1);
    un=povunion([cl,clx,cly,c]);
    obj=povdifference(s,un,povlook(red));
    writeln(obj);
    writeAxes();
  endfunction
\end{eulerudf}
\begin{eulercomment}
Fungsi povanaglyph () melakukan semua ini. Parameternya seperti di
povstart () dan povend () digabungkan.
\end{eulercomment}
\begin{eulerprompt}
>povanaglyph("myscene",zoom=4.5);
\end{eulerprompt}
\eulerimg{27}{images/Rofi'ah 'Aisy Zukhruf_23030630019_Matematika B_EMT4Plot3D-130.png}
\eulerheading{Mendefinisikan Objek Sendiri}
\begin{eulercomment}
Antarmuka povray Euler berisi banyak objek. Tetapi Anda tidak dibatasi
untuk ini. Anda dapat membuat objek sendiri, yang menggabungkan objek
lain, atau merupakan objek yang sama sekali baru.

Kami mendemonstrasikan torus. Perintah Povray untuk ini adalah
"torus". Jadi kami mengembalikan string dengan perintah ini dan
parameternya. Perhatikan bahwa torus selalu berpusat pada asalnya.
\end{eulercomment}
\begin{eulerprompt}
>function povdonat (r1,r2,look="") ...
\end{eulerprompt}
\begin{eulerudf}
    return "torus \{"+r1+","+r2+look+"\}";
  endfunction
\end{eulerudf}
\begin{eulercomment}
Here is our first torus.
\end{eulercomment}
\begin{eulerprompt}
>t1=povdonat(0.8,0.2)
\end{eulerprompt}
\begin{euleroutput}
  torus \{0.8,0.2\}
\end{euleroutput}
\begin{eulercomment}
Mari kita gunakan objek ini untuk membuat torus kedua, diterjemahkan
dan diputar.
\end{eulercomment}
\begin{eulerprompt}
>t2=povobject(t1,rotate=xrotate(90°),translate=[0.8,0,0])
\end{eulerprompt}
\begin{euleroutput}
  object \{ torus \{0.8,0.2\}
   rotate 90 *x 
   translate <0.8,0,0>
   \}
\end{euleroutput}
\begin{eulercomment}
Sekarang kami menempatkan objek ini ke dalam sebuah adegan. Untuk
tampilan, kami menggunakan Phong Shading.
\end{eulercomment}
\begin{eulerprompt}
>povstart(center=[0.4,0,0],angle=0°,zoom=3.8,aspect=1.5); ...
>writeln(povobject(t1,povlook(green,phong=1))); ...
>writeln(povobject(t2,povlook(green,phong=1))); ...
\end{eulerprompt}
\begin{eulerttcomment}
 >povend();
\end{eulerttcomment}
\begin{eulercomment}
memanggil program Povray. Namun, jika terjadi kesalahan, program
tersebut tidak menampilkan kesalahan. Oleh karena itu, Anda harus
menggunakan

\end{eulercomment}
\begin{eulerttcomment}
 >povend(<exit);
\end{eulerttcomment}
\begin{eulercomment}

jika ada yang tidak berhasil. Ini akan membiarkan jendela Povray
terbuka.


\textgreater{} povend ();

memanggil program Povray. Namun, jika terjadi kesalahan, itu tidak
menampilkan kesalahan. Karena itu Anda harus menggunakan

\end{eulercomment}
\begin{eulerttcomment}
 > povend (<exit);
\end{eulerttcomment}
\begin{eulercomment}

jika ada yang tidak berhasil. Ini akan membuat jendela Povray terbuka.
\end{eulercomment}
\begin{eulerprompt}
>povend(h=320,w=480);
\end{eulerprompt}
\eulerimg{18}{images/Rofi'ah 'Aisy Zukhruf_23030630019_Matematika B_EMT4Plot3D-131.png}
\begin{eulercomment}
Berikut adalah contoh yang lebih lengkap. Kami menyelesaikannya

\end{eulercomment}
\begin{eulerformula}
\[
Ax \le b, \quad x \ge 0, \quad c.x \to \text{Max.}
\]
\end{eulerformula}
\begin{eulercomment}
dan menunjukkan poin yang layak dan optimal dalam plot 3D.
\end{eulercomment}
\begin{eulerprompt}
>A=[10,8,4;5,6,8;6,3,2;9,5,6];
>b=[10,10,10,10]';
>c=[1,1,1];
\end{eulerprompt}
\begin{eulercomment}
Pertama, mari kita periksa, apakah contoh ini memiliki solusi.
\end{eulercomment}
\begin{eulerprompt}
>x=simplex(A,b,c,>max,>check)'
\end{eulerprompt}
\begin{euleroutput}
  [0,  1,  0.5]
\end{euleroutput}
\begin{eulercomment}
Yes, it has.

Next we define two objects. The first is the plane

\end{eulercomment}
\begin{eulerformula}
\[
a \cdot x \le b
\]
\end{eulerformula}
\begin{eulerprompt}
>function oneplane (a,b,look="") ...
\end{eulerprompt}
\begin{eulerudf}
    return povplane(a,b,look)
  endfunction
\end{eulerudf}
\begin{eulercomment}
Kemudian kami mendefinisikan perpotongan dari semua setengah spasi dan
sebuah kubus.
\end{eulercomment}
\begin{eulerprompt}
>function adm (A, b, r, look="") ...
\end{eulerprompt}
\begin{eulerudf}
    ol=[];
    loop 1 to rows(A); ol=ol|oneplane(A[#],b[#]); end;
    ol=ol|povbox([0,0,0],[r,r,r]);
    return povintersection(ol,look);
  endfunction
\end{eulerudf}
\begin{eulercomment}
We can now plot the scene.
\end{eulercomment}
\begin{eulerprompt}
>povstart(angle=120°,center=[0.5,0.5,0.5],zoom=3.5); ...
>writeln(adm(A,b,2,povlook(green,0.4))); ...
>writeAxes(0,1.3,0,1.6,0,1.5); ...
\end{eulerprompt}
\begin{eulercomment}
The following is a circle around the optimum.

--Terjemahan

Berikut ini adalah lingkaran di sekitar optimal.
\end{eulercomment}
\begin{eulerprompt}
>writeln(povintersection([povsphere(x,0.5),povplane(c,c.x')], ...
>  povlook(red,0.9)));
\end{eulerprompt}
\begin{eulercomment}
Dan ada kesalahan di arah optimal.
\end{eulercomment}
\begin{eulerprompt}
>writeln(povarrow(x,c*0.5,povlook(red)));
\end{eulerprompt}
\begin{eulercomment}
Kami menambahkan teks ke layar. Teks hanyalah objek 3D. Kita perlu
menempatkan dan memutarnya sesuai dengan pandangan kita.
\end{eulercomment}
\begin{eulerprompt}
>writeln(povtext("Linear Problem",[0,0.2,1.3],size=0.05,rotate=125°)); ...
>povend();
\end{eulerprompt}
\eulerimg{28}{images/Rofi'ah 'Aisy Zukhruf_23030630019_Matematika B_EMT4Plot3D-132.png}
\eulerheading{Contoh Lainnya}
\begin{eulercomment}
Anda dapat menemukan beberapa contoh Povray di Euler di file berikut.

See: Examples/Dandelin Spheres\\
See: Examples/Donat Math\\
See: Examples/Trefoil Knot\\
See: Examples/Optimization by Affine Scaling
\end{eulercomment}
\eulerheading{Latihan Soal}
\begin{eulercomment}
1. Buatlah objek torus dengan radius luar 1.0 dan radius dalam 0.3,
dan tempatkan objek ini di koordinat (1, 0, 0). Tampilkan objek ini
dengan Phong shading.

JAWAB :
\end{eulercomment}
\begin{eulerprompt}
>function povdonat(r1, r2, look="") ...
\end{eulerprompt}
\begin{eulerudf}
  $  return "torus \{" + r1 + "," + r2 + look + "\}";
  $endfunction
\end{eulerudf}
\begin{eulerprompt}
>t1 = povdonat(1.0, 0.3); ...
>povstart(center=[1, 0, 0], angle=0°, zoom=3); ...
>writeln(povobject(t1, povlook(green, phong=1))); ...
>povend():
\end{eulerprompt}
\eulerimg{28}{images/Rofi'ah 'Aisy Zukhruf_23030630019_Matematika B_EMT4Plot3D-133.png}
\eulerimg{27}{images/Rofi'ah 'Aisy Zukhruf_23030630019_Matematika B_EMT4Plot3D-134.png}
\begin{eulercomment}
2. Buatlah objek torus dengan radius luar 1.0 dan radius dalam 0.4,
dan tempatkan objek ini di koordinat (0, 0, 0).

JAWAB :
\end{eulercomment}
\begin{eulerprompt}
>function povdonat(r1, r2, look="") ...
\end{eulerprompt}
\begin{eulerudf}
  $  return "torus \{" + r1 + "," + r2 + look + "\}";
  $endfunction
\end{eulerudf}
\begin{eulerprompt}
>t1 = povdonat(1.0, 0.4);
>povstart(center=[0, 0, 0], angle=0°, zoom=3); ...
>writeln(povobject(t1, povlook(green))):
\end{eulerprompt}
\eulerimg{27}{images/Rofi'ah 'Aisy Zukhruf_23030630019_Matematika B_EMT4Plot3D-135.png}
\begin{eulercomment}
3. Buatlah dua objek kubus dengan ukuran 1 yang berada pada titik (0,
0, 0) dan (1, 1, 1). Gunakan warna biru untuk kubus pertama dan merah
untuk kubus kedua.

JAWAB :
\end{eulercomment}
\begin{eulerprompt}
>function povbox(center, size) ...
\end{eulerprompt}
\begin{eulerudf}
  $  return "box \{" + center + ", " + size + "\}";
  $endfunction
\end{eulerudf}
\begin{eulerprompt}
>cube1 = povbox([0, 0, 0], [1, 1, 1]);
\end{eulerprompt}
\begin{euleroutput}
  Need matrices or strings for |
  Try "trace errors" to inspect local variables after errors.
  povbox:
      return "box \{" + center + ", " + size + "\}";
  Error in:
  cube1 = povbox([0, 0, 0], [1, 1, 1]); ...
                                      ^
\end{euleroutput}
\begin{eulerprompt}
>cube2 = povbox([1, 1, 1], [1, 1, 1]);
\end{eulerprompt}
\begin{euleroutput}
  Need matrices or strings for |
  Try "trace errors" to inspect local variables after errors.
  povbox:
      return "box \{" + center + ", " + size + "\}";
  Error in:
  cube2 = povbox([1, 1, 1], [1, 1, 1]); ...
                                      ^
\end{euleroutput}
\begin{eulerprompt}
>povstart(center=[0.5, 0.5, 0.5], angle=45°, zoom=3); ...
>writeln(povobject(cube1, povlook(blue))); ...
>writeln(povobject(cube2, povlook(red))); ...
>povend()
\end{eulerprompt}
\begin{euleroutput}
  Variable or function cube1 not found.
  Error in:
  ... 0.5], angle=45°, zoom=3); writeln(povobject(cube1, povlook(blu ...
                                                       ^
\end{euleroutput}
\begin{eulercomment}
4. Buatlah objek bola dengan jari-jari 0.5 yang terletak di koordinat
(0, 0, 0) dan tampilkan dengan efek refleksi.

JAWAB :
\end{eulercomment}
\begin{eulerprompt}
>function povsphere(radius, look="") ...
\end{eulerprompt}
\begin{eulerudf}
  $  return "sphere \{" + radius + look + "\}";
  $endfunction
\end{eulerudf}
\begin{eulerprompt}
>ball = povsphere(0.5, povlook(green, reflection=0.8));
>povstart(center=[0, 0, 0], angle=0°, zoom=3);
>writeln(ball); 
>povend();
\end{eulerprompt}
\begin{euleroutput}
  Error : Povray error!
  
  Error generated by error() command
  
  povray:
      error("Povray error!");
  Try "trace errors" to inspect local variables after errors.
  povend:
      povray(file,w,h,aspect,exit); 
\end{euleroutput}
\begin{eulercomment}
5. Buatlah dua objek silinder dengan tinggi 1 dan jari-jari 0.3 yang
terletak pada posisi (0, 0, 0) dan (1, 0, 0). Tampilkan dengan warna
kuning dan ungu.

JAWAB :
\end{eulercomment}
\begin{eulerprompt}
>function povcylinder(radius, height, look="") ...
\end{eulerprompt}
\begin{eulerudf}
  
  $  return "cylinder \{" + radius + ", " + height + look + "\}";
  $endfunction
\end{eulerudf}
\begin{eulerprompt}
>cylinder1 = povcylinder(0.3, 1, povlook(yellow));
>cylinder2 = povcylinder(0.3, 1, povlook(purple));
\end{eulerprompt}
\begin{euleroutput}
  Variable or function purple not found.
  Error in:
  cylinder2 = povcylinder(0.3, 1, povlook(purple)); ...
                                                ^
\end{euleroutput}
\begin{eulerprompt}
>povstart(center=[0.5, 0, 0], angle=45°, zoom=3);
>writeln(povobject(cylinder1, povobject(translate=[0, 0, 0]))); 
\end{eulerprompt}
\begin{euleroutput}
  Function povobject needs at least one argument!
  Use: povobject (object: string \{, look: string, translate: none, rotate: none, scale: none\}) 
  Error in:
  ... vobject(cylinder1, povobject(translate=[0, 0, 0])));  ...
                                                       ^
\end{euleroutput}
\begin{eulerprompt}
>writeln(povobject(cylinder2, povobject(translate=[1, 0, 0]))); 
\end{eulerprompt}
\begin{euleroutput}
  Variable or function cylinder2 not found.
  Error in:
  writeln(povobject(cylinder2, povobject(translate=[1, 0, 0]))); ...
                             ^
\end{euleroutput}
\begin{eulerprompt}
>povend();
\end{eulerprompt}
\begin{euleroutput}
  Error : Povray error!
  
  Error generated by error() command
  
  povray:
      error("Povray error!");
  Try "trace errors" to inspect local variables after errors.
  povend:
      povray(file,w,h,aspect,exit); 
\end{euleroutput}
\end{eulernotebook}
\end{document}
